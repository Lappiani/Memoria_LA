\documentclass{article}
\usepackage{graphicx}% Required for inserting images
\usepackage[margin=1in]{geometry}
\usepackage{fancyhdr}
\usepackage{titling}
\usepackage{amsmath}
\usepackage{amsthm}
\usepackage{amssymb}
\usepackage{bm}
\usepackage{mdframed}
\usepackage{hyperref}


\pagestyle{fancy}
\fancyhf{}
\fancyhead[R]{Página \thepage}

%Para que los parrafos no tengan sangría.
\setlength{\parindent}{0pt}

%Titulo
\setlength{\droptitle}{-1.5cm}
\title{\textbf{Sin stocks}}
\date{}

%Ambientes
\newenvironment{boxedproof}
  {\begin{mdframed}\begin{proof}}
  {\end{proof}\end{mdframed}}


\begin{document}

\maketitle
\thispagestyle{fancy}
\vspace{-2cm}
\section{Desarrollo modelo}
El modelo del \textit{paper} es el siguiente:
\begin{gather}
    \min_{\bf{n}_{U}}\mathrm{E}\left[U(w_{U})\right] \label{objective}
\end{gather}
\begin{flalign}
    &\text{Sujeto a:}\nonumber&&\\
    &w_{U} = \boldsymbol{n^{\intercal}}_{U}\bm{x} + n^{f}_{U} \label{final wealth}&&\\
    &\Bar{w}_{U} = {\bm{n^{\intercal}}_{U}\bm{P}} + n^{f}_{U} P_{f}\label{initial wealth}
\end{flalign}
Donde:
\begin{itemize}
    \item $w_{U}:$ riqueza de fin de periodo para el inversionista sin restricciones.
    \item $\bm{n}_{U}$: vector que representa la cantidad de acciones que el inversionista tipo $U$ compra en cada uno de los $N$ activos riesgosos.
    \item $\bm{x}$: es el vector de pagos por acción en cada uno de los $N$ activos riesgosos.
    \item $n^{f}_{U}$: cantidad de bonos libres de riesgo en descuento con pagos unitarios que compra el inversionista tipo $U$.
    \item $\bm{P}$: vector de precios de los activos riesgosos.
    \item $P_{f}:$ precio del bono en descuento.
\end{itemize}
En cuanto a las restricciones, la ecuación \eqref{final wealth} corresponde a la riqueza final y la ecuación \eqref{initial wealth} a la riqueza inicial del inversionista $U$.\\

De \eqref{initial wealth} se puede concluir lo siguiente:
\begin{equation*}
    n^{f}_{U}= \frac{1}{P_{f}}\left(\Bar{w}_{U} - \bm{n^{\intercal}_{U}}\bm{P}\right)
\end{equation*}
Luego, reemplazándolo en \eqref{final wealth} se tiene lo siguiente.
\begin{equation}
    w_{U} = \bm{n^{\intercal}}_{U}\bm{x} + \Bar{w}_{U}\frac{1}{P_{f}} -\bm{n^{\intercal}}_{U}\underbrace{\frac{\bm{P}}{P_{f}}}_{\bm{p}} = \frac{\Bar{w}_{U}}{P_{f}} + \bm{n^{\intercal}}_{U}(\bm{x}-\bm{p}) \label{wealth cons.}
\end{equation}
La cual corresponde a la restricción de riqueza que se menciona en el \textit{paper}. Entonces, reemplazando \eqref{wealth cons.} en \eqref{objective} y derivando se tiene lo siguiente.
\begin{equation*}
    \begin{split}
        \frac{d\mathrm{E}}{dw_{U}} &= \mathrm{E}\left[U'(w_{U})(\bm{x}-\bm{p})\right]=0\;.
    \end{split}
\end{equation*}
Ahora dado que $\bm{x}\sim\mathcal{N}(\Bar{\bm{x}},\bm{\Sigma})$, aplicando la definición de covarianza tendremos lo siguiente,
\begin{equation*}
	\begin{split}
		\mathrm{E}\left[U'(w_{U})(\bm{x}-\bm{p})\right] &= E\left[(U'(w_U)-E\left[U'(w_U)\right])(\bm{x}-\bm{p}-E\left[\bm{x}-\bm{p}\right])\right] + E\left[U'(w_U)\right]E\left[\bm{x}-\bm{p}\right]\;,\\
		\mathrm{E}\left[U'(w_{U})(\bm{x}-\bm{p})\right] &= E\left[(U'(w_U)-E[U'(w_U)])(\bm{x}-\bm{\Bar{x}})\right]+ E\left[U'(w_U)\right](\bm{\Bar{x}}-\bm{p})\;,\\
		\mathrm{E}\left[U'(w_{U})(\bm{x}-\bm{p})\right] &= E\left[U'(w_U)(\bm{x}-\Bar{\bm{x}})-E[U'(w_U)](\bm{x}-\bm{\Bar{x}})\right]+ E\left[U'(w_U)\right](\bm{\Bar{x}}-\bm{p})\;,\\
		\mathrm{E}\left[U'(w_{U})(\bm{x}-\bm{p})\right] &= E\left[U'(w_U)(\bm{x}-\Bar{\bm{x}})\right]-E[U'(w_U)]E\left[\bm{x}-\bm{\Bar{x}}\right]+ E\left[U'(w_U)\right](\bm{\Bar{x}}-\bm{p})\;,\\
		\mathrm{E}\left[U'(w_{U})(\bm{x}-\bm{p})\right] &= E\left[U'(w_U)(\bm{x}-\bm{\Bar{x}})\right]+ E\left[U'(w_U)\right](\bm{\Bar{x}}-\bm{p})=0\;.
	\end{split}
\end{equation*}
Luego con esto podemos definir la siguiente igualdad,
\begin{equation*}
	-E\left[U'(w_U)(\bm{x}-\bm{\Bar{x}})\right]= E\left[U'(w_U)\right](\bm{\Bar{x}}-\bm{p})
\end{equation*}

\begin{boxedproof}[Aplicación lema de \textit{Stein}]
    Tenemos que $\bm{x}\sim\mathcal{N}\left(\bm{\Bar{x}},\bm{\Sigma}\right)$, 
    \begin{equation*}
    \begin{split}
        \mathrm{E}[U'(w_{U})(\bm{x}-\bm{\Bar{x}})] &= \bm{\Sigma}\mathrm{E}\left[ \frac{\partial U'(w_{U})}{\partial \bm{x}}\right]\\
    \end{split}
    \end{equation*}
    Tomando $w_{u} = \bm{x^{\intercal}}\bm{n}_{U} + n^{f}_{U}\bm{1}\Rightarrow \frac{\partial w_{U}}{\partial \bm{x}} = \bm{n}_{U}$, luego,
    \begin{equation*}
        \mathrm{E}[U'(w_{U})(\bm{x}-\bm{\Bar{x}})] = \bm{\Sigma}\mathrm{E}\left[U''(w_{u})\bm{n}_{U}\right] = \mathrm{E}\left[U''(w_{u})\right]\bm{\Sigma}\bm{n}_{U}
    \end{equation*}
\end{boxedproof}

Entonces, sustituyendo la aplicación del lema que se mostró,
\begin{equation}
	\begin{split}
		-E\left[U''(w_U)\right]\bm{\Sigma}\bm{n}_{U}&=E\left[U'(w_U)\right](\bm{\Bar{x}}-\bm{p})\;,\\
		\bm{\Bar{\bm{x}}}-\bm{p} &= \frac{-E\left[U''(w_U)\right]}{E\left[U'(w_U)\right]}\bm{\Sigma}\bm{n}_U\;,\\
		\bm{\Bar{\bm{x}}}-\bm{p} &= \theta_U\bm{\Sigma}\bm{n}_U\;.
	\end{split}
	\label{eq:unrestricted}
\end{equation}
Donde $\theta_{U}$ es similar a la aversión absoluta al riesgo, que depende en la riqueza inicial del inversionista $U$. Luego, se plantea el mismo modelo, pero esta vez se hace para un inversionista tipo $R$ (el inversionista restrictivo representativo), el cual escoge boicotear acciones ``\textit{sin}''.
\begin{gather}
    \min_{\bf{n}_{R}}\mathrm{E}\left[U(w_{R})\right] \label{objective R}
\end{gather}
\begin{flalign}
    &\text{Sujeto a:}\nonumber&&\\
    &w_{R} = \boldsymbol{n^{\intercal}}_{R}\bm{x} + n^{f}_{R} \label{final wealth R}&&\\
    &\Bar{w}_{R} = {\bm{n^{\intercal}}_{R}\bm{P}} + n^{f}_{R} P_{f}\label{initial wealth R}
\end{flalign}
Donde $\bm{n}_{R}$ es el vector del número de acciones que el inversionista $R$ compra en $N_{N}$ activos que no son del tipo ``\textit{sin}''. Entonces, siguiendo el mismo procedimiento se llega a lo siguiente.
\begin{equation}
    \theta_{R}\bm{\Sigma}_{N}\bm{n}_{R}=\Bar{\bm{x}}_{N} - \bm{p}_{N}\label{non_sin_res}
\end{equation}
Donde la matriz covarianza de los pagos de los activos se encuentra dividida en empresas \textit{sin} ($S$) y \textit{nonsin} ($N$).
\begin{equation}
    \bm{\Sigma} = \begin{bmatrix}
        \bm{\Sigma}_{N} & \bm{\Sigma}_{NS}\\
        \bm{\Sigma}_{SN} & \bm{\Sigma}_{S}
    \end{bmatrix}
    \label{covariance_matrix_gen}
\end{equation}
Donde $\bm{\Sigma}_{N}$ representa la matriz de covarianza de pagos de todos los activos que no han sido boicoteados y $\Bar{\bm{x}}_{N}$, $\bm{p}_{N}$ son los vectores de pagos promedio y precios, respectivamente, de los activos que no han sido boicoteados. \\

Luego, asumiendo que se tienen $q_{U}$ inversionistas del tipo $U$ y $q_{R}$ del tipo $R$, la demanda por activos puede ser obtenida y ser equivalente a la oferta exógena de acciones, $\Bar{\bm{n}} = \left(\Bar{\bm{n}}_{N},\Bar{\bm{n}}_{S}\right)^{\intercal}$, y a cero para el activo libre de riesgo, lo que da lugar a las condiciones para el equilibrio de mercado.
\begin{equation}
    \Bar{\bm{n}} = q_{U}\bm{n}_{U} + q_{R}\bm{n}_{R},\quad 0 = q_{U}n^{f}_{U} + q_{R}n^{f}_{R}\label{market eq.}
\end{equation}
Reordenando las ecuaciones \eqref{non_sin_res} y \eqref{eq:unrestricted} se llega a los siguiente.
\begin{equation*}
    \bm{n}_{U} = \left(\theta_{U}\bm{\Sigma}\right)^{-1}(\Bar{\bm{x}} - \bm{p}),\quad \bm{n}_{R} = \left(\theta_{R}\bm{\Sigma}_{R}\right)^{-1}(\Bar{\bm{x}}_{N} - \bm{p}_{N})\;. 
\end{equation*}
Notemos que podemos reescribir $\bm{n}_R$ de la siguiente forma,
\begin{equation*}
	\bm{n}_R = \theta^{-1}_{R}\begin{bmatrix}
		\bm{\Sigma}^{-1}_{N} & 0\\
		0 & 0
	\end{bmatrix}
	(\Bar{\bm{x}}-\bm{p}) = \begin{bmatrix}
		\bm{I}\\
		0
	\end{bmatrix}
	\left(\bm{\Sigma}_N\theta_R\right)^{-1}\begin{bmatrix}
		\bm{I} & 0
	\end{bmatrix}
	\left(\bm{\Bar{x}}-\bm{p}\right)\;.
\end{equation*}
Luego reemplazando nos queda que,
\begin{equation}
	\bm{\Bar{n}}=\left(\left(\bm{\Sigma}\theta_U/q_U\right)^{-1}+\begin{bmatrix}
		\bm{I}\\
		0
	\end{bmatrix}
	\left(\bm{\Sigma}_N \theta_R/q_R\right)^{-1}\begin{bmatrix}
		\bm{I} & 0
	\end{bmatrix}\right)\left(\bm{\Bar{\bm{x}}}-\bm{p}\right)\;.
	\label{eq:dem_exogena}
\end{equation}
De donde queremos despejar la expresión $\bm{\Bar{x}}-\bm{p}$, por lo que es necesario calcular la inversa de la expresión en paréntesis. Esto se hace mediante el uso de una identidad que dice lo siguiente, dadas las matrices $\bm{X}_1, \bm{X}_2, \bm{X}_3 \text{ y } \bm{X}_4$, con $\bm{X}_1$, $\bm{X}_4$ invertibles, se cumple lo siguiente.
\begin{equation}
	\left(\bm{X}^{-1}_1 + \bm{X}_2\bm{X}^{-1}_{4}\bm{X}_3\right)^{-1} = \bm{X}_1 + \bm{X}_1\bm{X}_2\left(\bm{X}_4 + \bm{X}_3\bm{X}_1\bm{X}_2\right)^{-1}\bm{X}_3\bm{X}_1\;.
	\label{eq:id_sodes}
\end{equation}
Por lo que, reemplazando los términos en \eqref{eq:id_sodes}, tenemos que,
\begin{equation*}
	\begin{split}
		& \left(\left(\bm{\Sigma}\theta_U/q_U\right)^{-1}+\begin{bmatrix}
			\bm{I}\\
			0
		\end{bmatrix}
		\left(\bm{\Sigma}_N \theta_R/q_R\right)^{-1}\begin{bmatrix}
			\bm{I} & 0
		\end{bmatrix}\right)^{-1}\\ 
		&= \bm{\Sigma}\theta_U/q_U -\bm{\Sigma}\theta_U/q_U\begin{bmatrix}
		\bm{I}\\
		0
		\end{bmatrix}
		\left(\bm{\Sigma}_N \theta_R/q_R + \begin{bmatrix}
			\bm{I} & 0
		\end{bmatrix}
		\bm{\Sigma}\theta_U/q_U\begin{bmatrix}
			\bm{I}\\
			0
		\end{bmatrix}
		\right)
		\begin{bmatrix}
			\bm{I} & 0
		\end{bmatrix}
		\bm{\Sigma}\theta_U/q_U\;,\\
		&= \bm{\Sigma}\theta_U/q_U - \bm{\Sigma}\theta_U/q_U\begin{bmatrix}
			\bm{I}\\
			0
		\end{bmatrix}
		\left(\bm{\Sigma}_N \theta_R/q_R + \bm{\Sigma}_N\theta_U/q_U\right)^{-1}\begin{bmatrix}
			\bm{I} & 0
		\end{bmatrix}
		\bm{\Sigma}\theta_U/q_U\;,\\
		&= \theta_U/q_U\left(\bm{\Sigma} - \frac{\theta_U/q_U}{\theta_U/q_U + \theta_R/q_R}\bm{\Sigma}\begin{bmatrix}
			\bm{\Sigma}^{-1}_{N} & 0\\
			0 & 0
		\end{bmatrix}
		\bm{\Sigma}
		\right)\;.
	\end{split}
\end{equation*}
Entonces, reemplazando la expresión en \eqref{eq:dem_exogena} queda lo siguiente,
\begin{equation}
	\begin{split}
		(\bm{\Bar{x}}-\bm{p})&=\theta_{U}/q_{U}\left(\bm{\Sigma} - \frac{\theta_U/q_U}{\theta_U/q_U + \theta_R/q_R}\bm{\Sigma}\begin{bmatrix}
			\bm{\Sigma}^{-1}_{N} & 0\\
			0 & 0
		\end{bmatrix}
		\bm{\Sigma}
		\right)\bm{\Bar{n}}\;,\\
		&=\theta_{U}/q_{U}\left(\bm{\Sigma} - \frac{\theta_U/q_U}{\theta_U/q_U + \theta_R/q_R}\bm{\Sigma}\begin{bmatrix}
			\bm{I} & \bm{\Sigma}^{-1}_{N}\bm{\Sigma}_{NS}\\
			0 & 0
		\end{bmatrix}
		\right)\bm{\Bar{n}}\;,\\
		&= \theta_{U}/q_{U}\left(\bm{\Sigma}\bm{\Bar{n}} - \frac{\theta_U/q_U}{\theta_U/q_U + \theta_R/q_R}\bm{\Sigma}\begin{bmatrix}
			\bm{\Bar{n}}_{N}+\bm{\Sigma}^{-1}_{N}\bm{\Sigma}_{NS}\bm{\Bar{n}}_{S}\\
			0 
		\end{bmatrix}
		\right)\;,\\
		&= \theta_{U}/q_{U}\left(\bm{\Sigma}\bm{\Bar{n}} -\frac{\theta_U/q_U}{\theta_U/q_U + \theta_R/q_R}\bm{\Sigma}\bm{\Bar{n}}+ \frac{\theta_U/q_U}{\theta_U/q_U + \theta_R/q_R}\bm{\Sigma}\begin{bmatrix}
			-\bm{\Sigma}^{-1}_{N}\bm{\Sigma}_{NS}\bm{\Bar{n}}_{S}\\
			\bm{\Bar{n}}_{S} 
		\end{bmatrix}
		\right)\;,\\
		&= \theta_{U}/q_{U}\left(\frac{\theta_R/q_R}{\theta_U/q_U + \theta_R/q_R}\bm{\Sigma}\bm{\Bar{n}}+ \frac{\theta_U/q_U}{\theta_U/q_U + \theta_R/q_R}\bm{\Sigma}\begin{bmatrix}
			-\bm{\Sigma}^{-1}_{N}\bm{\Sigma}_{NS}\bm{\Bar{n}}_{S}\\
			\bm{\Bar{n}}_{S} 
		\end{bmatrix}
		\right)\;,\\
		&= \left(\frac{1}{q_U/\theta_U + q_R/\theta_R}\bm{\Sigma}\bm{\Bar{n}}+ \frac{1}{q_U/\theta_U + q_R/\theta_R}\frac{q_{R}/\theta_R}{q_U/\theta_U}\bm{\Sigma}\begin{bmatrix}
			-\bm{\Sigma}^{-1}_{N}\bm{\Sigma}_{NS}\bm{\Bar{n}}_{S}\\
			\bm{\Bar{n}}_{S} 
		\end{bmatrix}
		\right)\;,\\
		&= \frac{1}{q_U\Bar{w}_U/\rho_U + q_R\Bar{w}_R/\rho_R}\bm{\Sigma}\bm{\Bar{n}}+ \frac{1}{q_U\Bar{w}_U/\rho_U + q_R\Bar{w}_R/\rho_R}\frac{q_{R}\Bar{w}_R/\rho_R}{q_U\Bar{w}_U/\rho_U}\bm{\Sigma}\begin{bmatrix}
			-\bm{\Sigma}^{-1}_{N}\bm{\Sigma}_{NS}\bm{\Bar{n}}_{S}\\
			\bm{\Bar{n}}_{S} 
		\end{bmatrix}
		\;,\\
		&=\gamma\bm{\Sigma}\bm{\Bar{n}} + \delta\bm{\Sigma}\bm{\Bar{n}}_B\;.
	\end{split}
	\label{eq:id-boicott}
\end{equation}
Donde $\bm{\Bar{n}}_B$ representa las posiciones de la cartera boicot. Ahora, hay que convertir la \eqref{eq:id-boicott} en una expresión para retornos esperados. Sabiendo que $P_f=1/(1+r_f)$ podemos definir,
\begin{equation*}
	(1+r^{s}_{i}) = \frac{x_i}{P_i} \Leftrightarrow x_i - \frac{P_i}{P_f}=P_i(1+r^{s}_{i}) - P_i(1+r_f)=P_i(r^{s}_{i}-r_f)\;.
\end{equation*}
Luego, definiendo el premio por riesgo como $r_i=r^{s}_{i}-r_{f}$, y dado que en \eqref{eq:id-boicott} la expresión a la izquierda de la igualdad está representada como promedio, se tiene que $\mu_{i}=\mu^{s}_{i}-r_{f}$. Además, como $1+r^{s}_{i}=x_i/P_i$, la matriz de covarianza para los pagos de los activos riesgosos $\bm{\Sigma}$ se puede representar en términos de los retornos de la siguiente forma $\sigma_{ij}=\Sigma_{ij}/P_{i}P_{j}$. Por ende, para un elemento de \eqref{eq:id-boicott} se tiene que,
\begin{equation}
	\begin{split}
		P_i \mu_i&=\gamma \Sigma_{im} + \delta\Sigma_{ib}\\
		\mu_{i} &= \gamma P_m \sigma_{im} + \delta P_b \sigma_{ib}
	\end{split}
	\label{eq:mean-returns}
\end{equation}
Donde $m$ representa el mercado, $P_m =q_m \Bar{w}_{M}=q_U \Bar{w}_{U} + q_R \Bar{w}_{R} $ es el costo del portafolio de mercado, y $P_b$ es el costo del portafolio boicot. Ahora, con \eqref{eq:mean-returns} podemos definir $\mu_{m}$ y $\mu_{b}$, que vendría siendo el retorno promedio del portafolio de mercado y del portafolio boicot, respectivamente.
\begin{equation*}
	\mu_{m} = \gamma P_m \sigma^{2}_{m} + \delta P_b \sigma_{mb} \quad; \quad \mu_{b} = \gamma P_m \sigma_{mb} + \delta P_b \sigma^{2}_{b}\;.
\end{equation*}
Solucionando este sistema de ecuaciones para $\gamma P_m$ y $\delta P_b$ se obtiene los siguiente,
\begin{equation*}
	\delta P_b = \frac{\sigma_{mb}\mu_{m}-\sigma^{2}_{m}\mu_{b}}{\sigma^{2}_{mb}-\sigma^{2}_{b}\sigma^{2}_{m}}\quad ; \quad \gamma P_m = \frac{\sigma_{mb}\mu_{b}-\sigma^{2}_{b}\mu_{m}}{\sigma^{2}_{mb}-\sigma^{2}_{b}\sigma^{2}_{m}}\;.
\end{equation*}
Reemplazando en \eqref{eq:mean-returns} queda lo siguiente,
\begin{equation}
	\begin{split}
		\mu_{i} &= \frac{\sigma^{2}_{m}\sigma_{ib}-\sigma_{mb}\sigma_{im}}{\sigma^{2}_{b}\sigma^{2}_{m}-\sigma^{2}_{mb}}\mu_{b} + \frac{\sigma^{2}_{b}\sigma_{im}-\sigma_{mb}\sigma_{ib}}{\sigma^{2}_{b}\sigma^{2}_{m}-\sigma^{2}_{mb}}\mu_{m}\;,\\
		&= \beta_{ib}\mu_{b} + \beta_{im}\mu_{m}\;.\\  
	\end{split}
\end{equation}
Ahora de \eqref{eq:id-boicott}, despejando $\bm{p}$ tenemos lo siguiente,
\begin{equation*}
	\bm{p} = \bm{\Bar{x}} - \left(\gamma \bm{\Sigma}\bm{\Bar{n}} + \delta \bm{\Sigma}\bm{\Bar{n}}_{B}\right)\;.
\end{equation*}
Luego, multiplicando por un vector de tenencias de una cartera $i$ nos que para un portafolio específico $i$,
\begin{equation}
	p_{i} = \bm{\Bar{n}}^{\intercal}_{i}\bm{p} = \Bar{x}_{i}-\gamma\Sigma_{im}-\delta\Sigma_{ib}\;.
\end{equation}
Dado que $q_R>0$ cuando existen inversionistas con restricciones, esto implica que $\delta>0$. Por lo tanto, el precio por riesgo de boicot es positivo. Cuanto mayor sea la covarianza de los pagos del portafolio $i$ con el factor de pago de boicot $\Sigma_{ib}=\bm{\Bar{n}}^{\intercal}{i}\bm{\Sigma}\bm{\Bar{n}}{B}$, menor será su precio en relación con el activo libre de riesgo, $p_{i}=P_{i}/P_{f}=P_{i}(1+r_{f})$. Esto a su vez resulta en un retorno promedio mayor para el portafolio $i$, dado por $\mu_{i}=(\bm{\Bar{n}}^{\intercal}_{i}\bm{\Bar{x}}/P_i)-(1/P_f)$.

Para derivar el premio por riesgo boicot, $\mu_{b}$, se construye el factor boicot $x_{b}-p_{b}=\bm{\Bar{n}}^{\intercal}_{B}(\bm{x}-\bm{p})$. Además aplicando el promedio,
\begin{equation}
	\begin{split}
		\Bar{x}_{b}-p_{b}&=\bm{\Bar{n}}^{\intercal}_{B}(\bm{\Bar{x}}-\bm{p})\\
		&= \begin{bmatrix}
			-\bm{\Bar{n}}^{\intercal}_{S}\bm{\Sigma}^{\intercal}_{NS}(\bm{\Sigma^{-1}_{N}})^{\intercal} & \bm{\Bar{n}}^{\intercal}_{S}
		\end{bmatrix}
		\left(\gamma\begin{bmatrix}
			\bm{\Sigma}_{N}\bm{\Bar{n}}_{N} + \bm{\Sigma}_{NS}\bm{\Bar{n}}_{S}\\
			\bm{\Sigma}_{SN}\bm{\Bar{n}}_{N} + \bm{\Sigma}_{S}\bm{\Bar{n}}_{S}
		\end{bmatrix}
		+\delta\begin{bmatrix}
			-\bm{\Sigma}_{NS}\bm{\Bar{n}}_{S} + \bm{\Sigma}_{NS}\bm{\Bar{n}}_{S}\\
			-\bm{\Sigma}_{SN}\bm{\Sigma}^{-1}_{N}\bm{\Sigma}_{NS}\bm{\Bar{n}}_{S} + \bm{\Sigma}_{S}\bm{\Bar{n}}_{S}
		\end{bmatrix}
		\right)\;,\\
		&= (\gamma+\delta)\bm{\Bar{n}}^{\intercal}_{S}(\bm{\Sigma}_{S} - \bm{\Sigma}_{SN}\bm{\Sigma}^{-1}_{N}\bm{\Sigma}_{NS})\bm{\Bar{n}}_{S}\;,\\
		&= (\gamma+\delta)\Sigma_{b}\;.
	\end{split}
\end{equation}
De donde es importante recordar que la matriz de covarianza por definición es simétrica, luego se cumple que $\bm{\Sigma}^{\intercal} = \bm{\Sigma}$. Ahora, para el retorno promedio,
\begin{equation}
	\begin{split}
		\mu_{b}&=\frac{\bm{\Bar{n}}^{\intercal}_{B}\bm{\bar{x}}}{P_{b}} - \frac{1}{P_f}\;,\\
		&= \frac{1}{P_b}\left(\Bar{x}_{b} - p_b + \bm{\Bar{n}}^{\intercal}_{B}\bm{p}\right)-\frac{1}{P_fz}\;,\\
		&=(\Bar{x}_{b}-p_b)\frac{P_f}{P_b}\frac{1}{P_f}\;,\\
		&=(\Bar{x}_{b}-p_b)\frac{1}{p_b P_f}\;,\\
		&=\frac{(1+r_f)(\gamma + \delta)\Sigma_{b}}{p_b}\;,\\
		\mu_{b}&= \frac{(\gamma + \delta)\Sigma_{b}(1+r_f)}{\Bar{x}_{b}-(\gamma + \delta)\Sigma_{b}}\;.
	\end{split}
	\label{e:mean-returns}
\end{equation}



%ANEXO
\newpage
\section{Anexo}
\subsection{Lema de \textit{Stein}}
\label{sec: stein}
Sea $X$ una variable aleatoria que distribuye normal con media $\mu$ y varianza $\sigma^2$. Sea $g$ una función para la cual existen $\mathrm{E}\left(g(X)(X-\mu)\right)$ y $\mathrm{E}\left(g'(X)\right)$. Entonces,
\begin{equation*}
    \mathrm{E}\left(g(X)(X-\mu)\right) =\sigma^{2}\mathrm{E}\left(g'(X)\right).
\end{equation*}
En general, suponiendo que $X$ e $Y$ tienen una distribución de probabilidad conjunta, entonces,
\begin{equation*}
    \mathrm{Cov}\left(g(X),Y\right) = \mathrm{Cov}(X,Y)\mathrm{E}(g'(X)).
\end{equation*}
Para un vector normal multivariado $(X_{1},\dots,X_{n})\sim \mathcal{N}(\bm{u},\bm{\Sigma})$, se cumple que,
\begin{equation}
    \mathrm{E}\left(g(\bm{X})(\bm{X}-\bm{\mu})\right) = \bm{\Sigma}\cdot \mathrm{E}\left(\nabla g(\bm{X})\right)\label{stein}
\end{equation}
\end{document}
