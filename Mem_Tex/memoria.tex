% Plantilla de memoria Universidad de los Andes
% para usuarios de pdfLaTeX.
%
% Jaime Cisternas, Santiago, junio de 2004, agosto 2010
%
% Necesita todos los archivos *.tex llamados en las
% sentencias \input{} (cap1, cap2,...)
% ademas de buc.bib y las figuras.
%
% El archivo setspace.sty permite especificar el espaciado
% entre las lineas. Los archivos newapa.sty y newapa.bst
% especifican el formato de la bibliografia.
%
% Para compilar :                    pdflatex memoria
% Para generar la bibliografia :     bibtex memoria, pdflatex memoria.tex (dos veces)

\documentclass[12pt]{report}

% El archivo siguiente contiene todos los detalles
% finos del formato de la memoria.
% No es necesario modificarlo.


% Definiciones basicas de LaTeX para la memoria
%
% No modificar las lineas a continuacion a menos
% de estar muy seguro de lo que se esta haciendo!!
%
% Jaime Cisternas, julio 2004, agosto 2010

% <jcisternas@uandes.cl> 

%%%%%%%%%%%%%%%%%%%%%%%%%%%%%%%%%%%%%%%%%%%%%%%%%%%%%%%%%%%%%%%%%%%%%

% Carga librerias utiles con simbolos y estructuras
\usepackage{amsfonts}
\usepackage{amssymb}
\usepackage{amsmath}   
\usepackage{amsthm}
\usepackage{bm}
\usepackage{latexsym}

% Libreria para ajustar espaciado entre lineas
\usepackage{setspace}
\usepackage{todonotes}

% Libreria con estilo APA para bibliografia
% \usepackage{core/newapa}
\usepackage[backend = biber, style=apa]{biblatex}
\addbibresource{bibliografia.bib}

% Si usted está utilizando un teclado en español, puede ser útil
% usar los paquetes a continuacion. De otra manera tendrá que generar las
% tildes y la letra ñ de forma especial.
% Estos caracteres no pueden ser utilizados en etiquetas ni tampoco en modo matemático
\usepackage[utf8]{inputenc}
% \usepackage[spanish]{babel}
% Librería para obtener fecha actual
\usepackage{datetime}

% Declaraciones especificas de pdfLaTeX
\usepackage[pdftex,
        colorlinks=false,         % true or false (for final version)
        urlcolor=rltblue,         % \href{...}{...} external (URL)
	    anchorcolor=rltbrightblue,
        filecolor=weben,          % \href*{...} local file
        linkcolor=webred,         % \ref{...} and \pageref{...}
        menucolor=webdarkblue,
        citecolor=webgreen,
        pdftitle={},              % INSERT YOUR TITLE HERE
        pdfauthor={},             % INSERT YOUR NAME HERE
        pdfsubject={},
        pdfkeywords={},
        pdfpagemode=None,
        bookmarksopen=true,
	plainpages=false]{hyperref}
	
\usepackage[]{graphicx}
\pdfcompresslevel=9
\pdfadjustspacing=1
% Definicion de colores
\usepackage{color}
\definecolor{rltbrightred}{rgb}{1,0,0}
\definecolor{rltred}{rgb}{0.75,0,0}
\definecolor{rltdarkred}{rgb}{0.5,0,0}
\definecolor{rltbrightgreen}{rgb}{0,0.75,0}
\definecolor{rltgreen}{rgb}{0,0.5,0}
\definecolor{rltdarkgreen}{rgb}{0,0,0.25}
\definecolor{rltbrightblue}{rgb}{0,0,1}
\definecolor{rltblue}{rgb}{0,0,0.75}
\definecolor{rltdarkblue}{rgb}{0,0,0.5}
\definecolor{webred}{rgb}{0.5,.25,0}
\definecolor{webblue}{rgb}{0,0,0.75}
\definecolor{webgreen}{rgb}{0,0.5,0}
\definecolor{webdarkblue}{rgb}{0,0,0.5}
\definecolor{webbrightgreen}{rgb}{0,0.75,0}
% fin de declaraciones especificas pdfLaTeX

% tamanio de pagina
\setlength{\oddsidemargin}{.5in}
\setlength{\evensidemargin}{.0in} % en caso de usar opcion 'twoside'
\setlength{\textwidth}{6in}
\setlength{\topmargin}{-.5in}
\setlength{\textheight}{9in}

\setlength{\parskip}{\baselineskip}

% Las siguientes definiciones pueden ser usadas en
% memorias mas matematicas
% \newtheorem{theorem}{Teorema}[section]
% \newtheorem{lemma}[theorem]{Lema}
% \newtheorem{corollary}[theorem]{Corolario}
% \newtheorem{proposition}[theorem]{Proposición}
% \newtheorem{definition}[theorem]{Definición}
% \newtheorem{claim}{Afirmación}
% \newtheorem{conjecture}[theorem]{Conjetura}
% \newtheorem{observation}[theorem]{Observación}
% \newtheorem{problem}[theorem]{Problema}

% Definicion de un ambiente para algoritmos en pseudo-codigo.
% Esta definicion puede ser mejorada.
\newtheorem{algorithm}{Algoritmo}[section]
\newcommand{\tab}{\hspace*{0.5 cm}}
% Palablas claves que deben aparecer en negrita
\newcommand{\bfWHILE}{\textbf{while~}}
\newcommand{\bfDO}{\textbf{do~}}
\newcommand{\bfEND}{\textbf{end~}}
\newcommand{\bfIF}{\textbf{if~}}
\newcommand{\bfTHEN}{\textbf{then~}}
\newcommand{\bfELSE}{\textbf{else~}}
\newcommand{\bfFOR}{\textbf{for~}}

% Nombres fijos de Latex pueden ser cambiados al Espaniol
% \renewcommand{\contentsname}{Índice General}
% \renewcommand{\chaptername}{Capítulo}
% \renewcommand{\appendixname}{Anexo}
% \renewcommand{\bibname}{Bibliografía}
% \renewcommand{\figurename}{Ilustración}
% \renewcommand{\tablename}{Tabla}
% \renewcommand{\indexname}{Índice}
% \renewcommand{\partname}{Parte}
% \renewcommand{\listfigurename}{Lista de Ilustraciones}
% \renewcommand{\listtablename}{Lista de Tablas}

% Para las enumeraciones usamos primero a,b,c,... y despues i, ii, iii,...
\renewcommand{\labelenumi}{\alph{enumi})}
\renewcommand{\labelenumii}{\roman{enumii})}
\renewcommand{\labelenumiii}{-}
\renewcommand{\labelenumiv}{-}

% Para ser consistentes en el uso de abreviaciones!
%\newcommand{\Chp}{Cap\'{\i}tulo}
%\newcommand{\Chps}{Cap\'{\i}tulos}
%\newcommand{\Sec}{Sec.} % \S
%\newcommand{\Secs}{Secs.} % \S\S
%\newcommand{\SSec}{Subsec.}
%\newcommand{\Fig}{Fig.}
%\newcommand{\Figs}{Figs.}
%\newcommand{\Eqn}{Ecn.}
%\newcommand{\Eqns}{Ecns.}

% Muestra en pantalla los nombres de los archivos usados
\listfiles

% Previene una cierta senial de ``warning: contentsline with no destination''
\newcounter{dummy}


%%%%%%%%%%%%%%%%%%%%%%%%%%%%%%%%%%%%%%%%%%%%%%%%%%%%%%%%%%%%%%%%%%%%%%%%

% LAS SIGUIENTES LINEAS DEBEN SER MODIFICADAS POR EL AUTOR DE LA MEMORIA
% SI EL TITULO ES MUY LARGO PUEDE SER CORTADO CON \linebreak

\newcommand{\nombreautor}{Luca Appiani Caro}
\newcommand{\mes}{\monthname}
\newcommand{\anio}{\the\year}
\newcommand{\titulo}{Formulation of an asset pricing model for cryptocurrencies}
\newcommand{\nombreprofuno}{Javier Mella}
\newcommand{\nombreprofdos}{Name of the Dean's Representative}
\newcommand{\nombreproftres}{Name of the Visiting Professor}
\newcommand{\codigo}{ING-IN-001/11}

% LAS SIGUIENTES DEFINICIONES TAMBIEN SON NECESARIAS.
% ES FUNDAMENTAL QUE EL TEXTO ESTE ENCERRADO ENTRE
% PARENTESIS DE LLAVE.

\newcommand{\resumen}{
Cryptocurrency has gained significant popularity in recent years due to various factors, including attractive returns and new investment opportunities. Additionally, the technologies underlying these assets are highly interesting to investors worldwide. Therefore, research aimed at identifying the factors driving cryptocurrency returns can be highly beneficial for both individual investors and financial institutions, which currently hold substantial investments in this type of assets.

The main objective of this dissertation is to create and confirm a Portfolio Markowitz-inspired model to evaluate cryptocurrency returns in order to improve comprehension and assist in making informed investment choices.

The methodology is comprised of five main steps. First, the mathematical formulation of the model was done, which was inspired by the research of \parencite{luo2017social}. The procedure is similar, although there were some steps that were not detailed in the original paper, but in this case the interpretation of various variables is changed, in order to fit the cryptocurrency focus. Then, a dataset of returns from cryptocurrencies was built, and to ensure the validity of said returns, some statistical tests were conducted. 

Following this, statistical tests were conducted, including Fama-MacBeth regressions and the Gibbons, Ross, and Shanken (GRS) test, to evaluate the model's capacity and validity in explaining the cross-section of cryptocurrency returns. Subsequently, conclusions and recommendations were drawn from these tests.

The most important results include the model passing the GRS test at a 5\% significance level, but not at 1\%. The Fama-MacBeth regressions yielded poor results, with factors not statistically significant and an average R-squared of 33\%. Principal conclusions indicate that while theory-based models may explain asset returns, empirical models with more factors outperform them when tested with real data.
}

\newcommand{\agradecimientos}{
% Muchas gracias a todas aquellas personas que quizás
% sin saberlo y sin mala voluntad de su parte
% colaboraron a la finalización de esta investigación... 

%En esta página se puede nombrar a toda clase de personas,
%amigos y parientes, sin olvidar al profesor guía y
%a la secretaria de la Facultad.
}

\newcommand{\dedicatoria}{\it
I dedicate this work to ...
% mis papás, a mis abuelitos,
% a mi nana y a mi tía del Kinder.
}

%%%%%%%%%%%%%%%%%%%%%%%%%%%%%%%%%%%%%%%%%%%%%%%%%%%%%%%%%%%%%%%%%%%%%

% Por favor coloque todas las definiciones de simbolos a continuacion
% y no en los archivos de capitulos. Esto evitara la  existencia de
% multiples definiciones para las mismas palabras.

\newcommand{\mydef}
	{\stackrel{\mathrm{def}}{=}}
\newcommand{\e}
	{\hbox{\large{e}}}
\newcommand{\mi}
	{\hbox{\large{i}}}
\newcommand{\RR}{\mathbb{R}}
\newcommand{\dd}{\mathrm{d}}
\newcommand{\refname}{References}
%%%%%%%%%%%%%%%%%%%%%%%%%%%%%%%%%%%%%%%%%%%%%%%%%%%%%%%%%%%%%%%%%%%%%%
\begin{document}
\label{start}

% Genera las paginas de titulo, copyright, resumen,
% agradecimientos, dedicatoria, indices,...

% No es necesario modificar

% Paginas de titulo, derechos de autor y otras mas.

% Jaime Cisternas, julio 2004, agosto 2010

% declaracion especifica de pdfLaTeX
% permite incluir figuras con \includegraphics[...]{figurename}
\DeclareGraphicsExtensions{.jpg,.pdf,.mps,.png}

%%%%%%%%%%%%%%%%%%%%%%%%%%%%%%%%%%%%%%%%%%%%%%%%%%%%%%%%%%%%%%%%

  \pagenumbering{roman}
% \setcounter{page}{1}

% Escribe la pagina de titulo, incluyendo el autor y la facultad

  \thispagestyle{empty}
  \textsc{
  \vspace*{0cm}
  \begin{center}
    \Large
     Universidad de los Andes \\
     School of Engineering and Applied Sciences\\
  \end{center}
     \vspace{1cm}
  \begin{center}
     \includegraphics[angle=0,height=5cm]{logos/Logo-UANDES.png}
  \end{center}
     \vspace{1cm}
  \begin{center}
     \Large
     \titulo
  \end{center}
  \vspace{1.25cm}
  \begin{center}
    \Large
    \nombreautor
  \end{center}
  \vspace{1.25cm}
  \begin{center}
    Dissertation for the degree of \\
    Civil Industrial Engineer \\
    \vspace{1cm}
    Guiding teacher: \nombreprofuno \\
    \vspace{0.5cm}
    \codigo \\
    \vspace{0.5cm}
    Santiago, \mes\ \anio
  \end{center}
  }

%%%%%%%%%%%%%%%%%%%%%%%%%%%%%%%%%%%%%%%%%%

% Escribe la segunda pagina de titulo para ser firmada por la comision

\cleardoublepage
\thispagestyle{empty}

\begin{center}

\vspace*{2cm}
\parbox{10cm}{
\noindent
I certify that I have read this dissertation and that, in my opinion, its scope and quality are fully adequate to be considered a dissertation leading to the title of Engineer.
\vspace{1cm}

\hfill
\begin{tabular}{c}
\hspace{8cm} \\
\hline
\nombreprofuno \\
(Guiding Professor)
\end{tabular}

\vspace*{1.5cm}

\noindent
I certify that I have read this dissertation and that, in my opinion, its scope and quality are fully adequate to be considered a dissertation leading to the title of Engineer.
\vspace{0.75cm}

\hfill
\begin{tabular}{c}
\hspace{8cm} \\
\hline
\nombreprofdos
\end{tabular}

\vspace*{1.5cm}

\noindent
I certify that I have read this dissertation and that, in my opinion, its scope and quality are fully adequate to be considered a dissertation leading to the title of Engineer.
\vspace{0.75cm}

\hfill
\begin{tabular}{c}
\hspace{8cm} \\
\hline
\nombreproftres
\end{tabular}
}

\end{center}

%%%%%%%%%%%%%%%%%%%%%%%%%%%%%%%%%%%%%%%%%%

% Espaciado entre lineas
  \onehalfspacing
  %\doublespacing

% Hace pagina de derechos de autor

  \cleardoublepage
  \thispagestyle{empty}
  %\vspace*{0in}
  \begin{center}
    \copyright\ \nombreautor\ \anio \\
    All rights reserved.
  \end{center}

%%%%%%%%%%%%%%%%%%%%%%%%%%%%%%%%%%%%%%%%%%%%%%%%%%%%%%%%%%%%%%%%%%%%%%

% Genera resumen leyendo definicion de resumen

  \cleardoublepage
  \refstepcounter{dummy} \addcontentsline{toc}{section}{Summary}
  \begin{center} \Large \textbf{Summary} \end{center}
  \resumen

% Genera agradecimientos leyendo definicion de agradecimientos

  \cleardoublepage \addcontentsline{toc}{section}{Acknowledgements}
  \begin{center} \Large \textbf{Acknowledgements} \end{center}
  \agradecimientos

% Genera dedicatoria

  \cleardoublepage \vspace*{1.5in}
  \begin{flushright} \dedicatoria \end{flushright}

% Produce indice general, listas de figuras y de tablas

  \cleardoublepage
  \tableofcontents
  \cleardoublepage
  \listoffigures
  \cleardoublepage
  \listoftables
  \cleardoublepage

  \normalsize
  \pagenumbering{arabic}

%%%%%%%%%%%%%%%%%%%%%%%%%%%%%%%%%%%%%%%%%%%%%%%%%%%%%%%%%%%

%\listoftodos
%%%%%%%%%%%%%%%%%%%%%%%%%%%%%%%%%%%%%%%%%%%%%%%%%%%%%%%%%%%%%%%%%%

% cap1.tex
\chapter{Introduction}
\label{c1} % la etiqueta para referencias
The cryptocurrency world is a very intriguing one, the high volatility and technologies the assets that form part of this market support, provoke a lot of interest. The fact this type of assets are decentralized, have a lack of regulatory oversight, and operate on a global scale, pose significant challenges for investors and financial institutions. 

Nonetheless, there has been a surge in investment options related to this asset class. Said growth is driven by several factors which include: the increasing demand from investors for exposure to cryptocurrencies, the emergence of new technologies, and the growing interest from institutional investors. It is crucial to highlight that this popularity extends not only to the asset class as a whole but also to the cryptocurrencies themselves.

With the increasing demand for effective models that can analyze and forecast the returns of cryptocurrencies, the popularity of these digital assets also continues to grow. Current empirical models in the literature offer valuable insights, yet frequently have a base of solid theoretical basis. Integrating finance and economics theories into cryptocurrency modeling can help grasp the factors influencing cryptocurrency returns and enhance the precision of prediction models.

This research plans to fill the gap in the literature by creating a model based on Portfolio Markowitz theory to improve the analysis of cryptocurrency returns. This model can offer investors and financial institutions important information on how to build portfolios, manage risks, and make investment decisions in the quickly changing cryptocurrency market. In the end, the goal of this study is to improve the comprehension of how cryptocurrency returns work and set the stage for smarter investment decisions in this developing asset category.

  
\cleardoublepage

\chapter{Theoretical Framework}
\label{c2}
\section{Literature review}
Due to the rising popularity in recent years of cryptocurrency, there has been much research related to digital currency, from which the field of asset pricing is no exception. The latter because there is a growing interest related to the study of the factors that affect the returns of this type of assets, which certainly translates into a lot of studies whose objective is the previously mentioned. While the research topics may seem similar, it is important to note that this allows for a comprehensive categorization of the studies, despite the broadness of the related research.

\subsection{Empirical Studies}
The first group corresponds to the empirical studies that test for the performance of widely accepted asset pricing models such as the CAPM (\cite{sharpe1964}, \cite{Lintner1965} and \cite{mossin1966equilibrium}), FF3 \parencite{fama1993}, FF5 \parencite{fama2015}, \parencite{carhart1997}, among others. The methodology is based on the recollection of data related to returns on a specific set of cryptocurrencies in a particular period to calculate the factors of the models mentioned previously. Due to the significant amount of investigation that follows said framework, there are also many studies that, in addition to the steps mentioned previously, complement with techniques that help understand better the underlying phenomena. For a better grasp of these groups of studies, some will be discussed that will most definitely aid the current investigation.

The first study included in the group of empirical studies is the one done by \parencite{gregoriou2019cryptocurrencies}. In this investigation they demonstrate that investors obtain abnormal excess returns on the London Stock Exchange from 2014 to 2017. The main reason behind this was because of earlier studies, like \parencite{bariviera2017inefficiency}, that found evidence of inefficiency and lack of regulation related to the cryptocurrency market. The data used corresponds to daily returns of all London Stock Exchange listed securities from the years 2014-2017, where they conclude that by applying CAPM, FF3, Carhart, and FF5, investors do indeed obtain excess returns by speculating in cryptocurrencies, suggesting that they are inefficient. While this dissertation primarily does not explore into the efficiency of cryptocurrency markets, the insights from \parencite{gregoriou2019cryptocurrencies} underline the broad applicability and versatility of such studies.

Another study is the one done by \parencite{liu2022common}, where they find that there are three factors that capture the cross-sectional expected cryptocurrency returns. Despite not forming part of the core of the investigation, this study mentions a relevant aspect corresponding to the different opinions people have related to cryptocurrency; they say there are two views about the related market. The first one says that all coins represent bubbles and fraud. On the other hand, the second states technology behind said markets may become an important innovation and that at least some coins may become assets that represent a stake in the future of the related technology \parencite{liu2022common}. 

With the current information of cryptocurrency markets it is difficult to establish right from wrong with respect to those opinions. However, either way, empirical studies like \parencite{liu2022common} contribute largely to understand the factors that better explain the returns of corresponding assets. Regarding the research itself, the factors studied were cryptocurrency size, momentum, volume, and volatility. It is important to mention that the study focuses only on those market-factors, because financial and accounting data\footnote{Referring to information related to a company's performance, revenue, expenses, financial statements, which are crucial for assessing a company's financial position.} was not available for the cross-section of the coins that were analyzed in the data.

Regarding the conclusions drawn from \parencite{liu2022common}, there are several to consider. Firstly, size and momentum factors well capture the cross-section of cryptocurrency returns. Furthermore, a three-factor model can be constructed using market information that is successful in pricing the strategies in the cryptocurrency market. A number of theoretical explanations are drawn for the factors. In relation to the cryptocurrency size premium, which refers to the phenomenon where the average returns of small firms are higher than those of large firms \parencite{song2023size}. Said effect can be applied to cryptocurrencies. The cryptocurrency size factor relates to the liquidity effect, which encompasses the ease, speed, and affordability that an investor can trade a certain asset \parencite{hasan2022liquidity}. Secondly, they find some evidence that the size premium is consistent with a mechanism proposed by cryptocurrency theories\footnote{Some of them include \parencite{NBERw26816}, \parencite{Prat2019FundamentalPO}, and \parencite{hhaa089}.}: the trade-off between capital gains and the convenience yield\footnote{According to \parencite{Hull_deriv} it corresponds to the benefits from holding the physical asset.}.

As to momentum, the conclusions show that they are in line with the investor overreaction channel, indicating the tendency of investors to react disproportionately to new information, which in turn causes the price of cryptocurrency to swing more than it should according to its intrinsic value \parencite{10.1371/journal.pone.0264522}.

Continuing the line of empirical validation studies, \parencite{thoma2020prospect} investigated whether an investing strategy modeled by Cumulative Prospect Theory (CPT) leads to a risk-adjusted outperformance, based on different factor models which include the \parencite{fama1993}. Cumulative Prospect theory is a model proposed by \parencite{kahneman1979prospect}, fits well in modeling how investors inform themselves about a certain cryptocurrency since they usually look at the price chart and then mentally represent a historical return distribution. 

So, according to \parencite{thoma2020prospect}, by looking at the price chart of cryptocurrency, investors evaluate the skewness\footnote{Measure of the asymmetry of a distribution.} and evaluate the asset as a gamble, similar to lottery. The conclusions imply that cryptocurrency holders choose high prospect theory values over low values\footnote{It is important to note that high or low prospect theory values refers to the ones that are obtained in the expression used to calculate the respective factor used in the study, that is implemented in the factor models used in the investigation.}, with investors generally favoring the latter. Due to this predilection, cryptocurrencies with high prospect theory values are overbought, which reduces future gains. Cryptocurrencies with low prospect theory values on the other hand, are less likely to be overbought and might result in larger future returns. While the previous study presented did not place significant emphasis on the regression models themselves, it uses those models to complement the main model of the investigation, which was the prospect theory model.

Another approach to the asset pricing of cryptocurrency is the one taken by \parencite{HAYES20171308}, where a regression model is estimated using cost of production factors, rather than the usual market factors that comprise the most popular asset pricing models. Concerning the factors, \parencite{HAYES20171308} concludes that more than 84\% of value formation can be explained by three variables: computational power (as a representative for mining difficulty), rate of coin production, and the relative hardness of the mining algorithm employed.

\parencite{Shen2020} also followed a similar framework to the ones already mentioned. This study proposes a three factor pricing model, consisting of market, size and reversal factors. This model is compared with cryptocurrency-CAPM or C-CAPM, which uses only excess market returns to explain returns of cryptocurrency portfolios. As to the conclusions, the three-factor model based on the three factors already mentioned, has a better performance than the C-CAPM at explaining the cryptocurrency returns.

The research carried out by \parencite{Erfanian2022} provides another way of looking at the asset pricing of cryptocurrencies, while maintaining, to some extent, the empirical studies framework mentioned in the beginning of this literature review. They apply a series of machine learning approaches to investigate whether macroeconomic, microeconomic, technical, and blockchain indicators based on economic theories can predict bitcoin prices. Regarding the factor-based conclusions, based on a multilinear regression, the most significant long-term predictors were those of a macroeconomic nature, as well as blockchain information. Moreover, the empirical results showed that SVR (Support Vector Regressions) is the best machine learning model, and the effectiveness of feature selection techniques varied, with no clear winner emerging. Thus, providing evidence of the superiority of machine learning models in comparison to traditional methods for Bitcoin price prediction that use empirical research.

\parencite{GROBYS20196} investigate about the popular momentum strategy implemented in the cryptocurrency market. Although there is no use of the more popular asset pricing models mentioned in the beginning \parencite{sharpe1964}, \parencite{fama1993}, \parencite{fama2015}, and \parencite{carhart1997}, in this case a time series approach is taken, that uses the return of a security over the past months to determine the investor position on said security in the following month. They do not find any significant evidence as to relevant momentum payoffs in the cryptocurrency market.

The research done by \parencite{CAI2024107052} uses salience theory of choice under risk to show that investor behavior drives cross-sectional cryptocurrency returns. The reason being that headlines have significantly influenced the crypto asset class, sparking investor fear of missing out on the ``crypto rush''. A salience payoff refers to a payoff that stands out from the average, which under the context of salience theory, draws the attention of the investor. To examine salience payoffs \parencite{CAI2024107052} construct a salience measure, which measures the difference between salience and equally weighted returns during a specific time period, weekly or monthly. To construct the ST measure, they follow the study of \parencite{COSEMANS2021460}. The empirical study of \parencite{CAI2024107052} contains two parts; the study of the predictability of ST on cross-sectional crypto returns, and the investigation of the viability of ST as a cross-sectional pricing factor. 

They conclude that given the asset class lacks fundamentals and has a concentrated clientele, the ST effect documented in the study is the strongest in the literature. In addition, they mention that ST is much more relevant for emerging assets that have high uncertainties. However, as the crypto market becomes more mainstream and attracts more institutional investors, the ST effect may lose its relevancy in explaining the return dynamics in the crypto market.  

Due to the extensive amount of research related to the empirical studies, a final investigation will be presented, but it is important to mention that there are much more variants of this type of studies. \parencite{Long2020} research the cross-sectional seasonality anomaly in cryptocurrency markets. Said anomaly suggests that assets with highest (lowest) average same-calendar month return tend to overperform (underperform) in the future. In simpler terms, if an investor plans to invest on a Monday, she or he should check which assets delivered the highest returns on Mondays in the past. The models used in this case include CAPM and FF3. As to the conclusions, results demonstrate that there is a strong and sizable seasonality phenomena. However, they emphasize a limitation of their study relating the short sample period.

\subsection{Theoretic Models}
Now, concerning the second category of studies, they correspond to theoretic models or models that are derived from a theoretical framework. It is important to note that, unlike the empirical validation research, the quantity of theoretical models is much less. Particularly, studies focusing on cryptocurrencies are notably scarce. In despite of said shortage, one related study was found.

\parencite{koutmos2021intertemporal} developed an intertemporal regime-switching asset pricing model characterized by heterogeneous agents that have different expectations in relation to the volatility of the prices of bitcoin. The fact that models are intertemporal, refers to the fact that the models take into account changes in market conditions and risks over time; and as to the regime-switching part, this means that said models can switch between different states or ``regimes", that could represent market conditions. Regarding the agents, there are three: mean-variance optimizers, speculators, and fundamentalists. Although the derivation of the model in this research does not come from a mathematical formulation, like the derivation of CAPM, it is interesting to review nevertheless. 

Through the definition of these agents, they formulate a way to represent the demand for bitcoins for each one. Then, assuming the market is only composed of said agents, they develop an asset pricing model. Finally, regarding the conclusions, one of them was that due to the special characteristics of bitcoin investors in terms of risk aversion, the fact that economic variables appear to not explain a significant part of returns is not much of a surprise. As to the models themselves, they manage to estimate the impacts of different types of investors during low and high bitcoin price volatility regimes. 

Lastly, the research done by \parencite{Bennett2023}, although it does not fit into any of the two groups of studies proposed in this literature review, it provides an interesting view about different behavioral finance aspects that apply to decentralized finance\footnote{Emerging financial technology, in which cryptocurrency could be considered.}. They mention that asset pricing in rapidly evolving markets is better explained through behavioral finance, rather than through traditional finance theory. Factors like investor attention, sentiment, heuristics and biases, and network effects interact to form a highly volatile and dynamic market \parencite{Bennett2023}. A particularly compelling aspect about said research, is that presents a theoretical model of behavioral finance applications for asset pricing models related to decentralized finance, that could be taken into account when an initial proposal of factors is made.
%\todo[inline]{This type of studies is on parenthesis. (on the paragraph above)} 

\subsection{Initial proposal of factors}
Having reviewed the related bibliography, determinant factors will now be proposed, which could be part of the mathematical formulation for the derivation of future models. It is important to note that the factors mentioned correspond to a preliminary proposal, and the specific way of how they could be included in the formulation of the models will not be addressed in this section.

Despite the fact that \parencite{GROBYS20196} do not find significant evidence as to relevant momentum payoffs, they evaluate only utilizing said factor as an investment strategy, but that does not mean that it does not explain the variability of cryptocurrency returns. So, in line with the conclusions outlined in the research of \parencite{liu2022common}, which does find an importance on momentum, the first factors to take into consideration are the momentum related. In order to provide greater insight, said factors are in relation to past returns (i.e. past one week returns), however the temporal aspects of said elements should be evaluated, to determine which alternative leads to better results. This type of factors could help model the behavioral finance side of the cryptocurrency market.

Furthermore, following the research of \parencite{liu2022common}, size related factors should also be studied. The inclusion of this type of variables in to a theoretic formulation could not be so straightforward, but it is important to take them into consideration because of their significance in explaining the returns of cryptocurrencies. 

Other aspects that could be accounted for correspond to representing different type of investors. In this case, the types of investors could be selected according to different characteristics, like for instance, introducing different levels of risk aversion.

Despite the fact that behavioral finance applications could be seen as endless, in terms of the different factors that could be derived from this area. Following an approach similar to the last presented research \parencite{Bennett2023} in the literature review, it would be interesting to study the viability of incorporating some factors that are of behavioral nature. Some alternatives could be: investor sentiment, investor psychological biases, or movement of other assets like commodities or stocks.

Finally, though there might be a great variety of factors that could be added to the mathematical formulation, the aggregation of them does not ensure that the model derived from said problem will explain a significant portion of the variation of the returns of cryptocurrencies. That is the reason why it is important to study whether the inclusion of a factor, significantly enhances the explanatory capacity of the model.

\cleardoublepage

\chapter{Objectives and Methodological Framework}
\label{c3}
\section{Objectives}
\label{c31}

By distinguishing between general and specific objectives and ensuring they are concrete, verifiable, and free from methodological details, these objectives should effectively guide your dissertation research and provide clear benchmarks for evaluation upon completion.
\subsection{General Objective}
Create and confirm a Portfolio Markowitz-inspired model to evaluate cryptocurrency returns in order to improve comprehension and assist in making informed investment choices within the cryptocurrency sector.
\subsection{Specific Objectives}
\begin{enumerate}
	\item[1.] Develop a theoretical framework based on Portfolio Markowitz theory, which includes mathematical equations and fundamental principles, to support the creation of a model for analyzing cryptocurrency returns.
	\item[2.] Gather and prepare necessary datasets on cryptocurrency returns, ensuring the data is accurate and appropriate for testing and validating the subject model.
	\item[3.] Conduct an empirical investigation following the methodology outlined by \cite{fama1993} to evaluate the predictive capacity and robustness of the proposed model in capturing the dynamics of cryptocurrency returns.
	\item[4.] Develop statistical analyses, including the GRS hypothesis test and mean adjusted R-squared examination, accounting for the signs of mean regression coefficients, to evaluate the model's explanatory power and identify potential areas of improvement.
	\item[5.] Combine results from model testing and statistical analysis to determine how well the model explains cryptocurrency return variations and its usefulness in shaping portfolio construction and risk management strategies.
	\item[6.] Present suggestions for future research paths and practical applications for investors and financial institutions utilizing the knowledge obtained from the constructed model and empirical studies.
\end{enumerate}
\section{Methodological Framework}
In order to solve the literature gap that was mentioned on chapter \ref{c1}, a description of the methodological framework will be proposed in order to achieve the aforementioned general objective, and the ones established in section \ref{c31}.

\subsection{Data Retrieval}
A crucial part of the investigation is the data that will be used to do the related tests in order to check the validity and overall performance of the model. Due to the nature of this investigation and the accessibility, the data provider that was be selected is \textit{Yahoo Finance}.

\textit{Yahoo Finance} has an integrated library in \textit{python}, that can be used to retrieve a wide range of data, from prices to financial ratios related to specific companies, and other market data. It has data from about ten thousand cryptocurrencies, but the only problem is that the ``symbols''\footnote{Referring to the form cryptocurrencies are normally presented in exchanges, for example BTC-USD, which corresponds to the price of Bitcoin in US dollars.} can not be retrieved directly from said library. In order to complete this task \textit{Yahooquery} was used.

\textit{Yahooquery} is a python interface to unofficial \textit{Yahoo Finance} API endpoints. So in this case, the endpoint related to cryptocurrency symbols ordered by market capitalization was used to retrieve said cryptocurrencies. The maximum amount of cryptocurrencies that this interface allowed to retrieve was 250. The combination of these tools allowed to retrieve data of prices from 2014 onward of 250 cryptocurrencies ordered by intraday market capitalization\footnote{Market value of a cryptocurrency's stock at any given point during the trading day.}.

\subsection{Mathematical Formulation}
Another part of the investigation that is essential is the derivation of the model. The general idea of the mathematical formulation comes from the derivation of the Capital Asset Pricing Model. Although the traditional optimization problem minimizes the variance of the portfolio, in this case an alternative approach that is also used will be taken.

Considering this scenario, the objective will be the utility function of a certain type of investor, which depends on the terminal wealth of said individual. The idea is to maximize this utility function subject to two constraints related to the initial and final wealth of the investor.

Through the development of this theoretical formulation a formal model can be derived that explains the cross-section of returns of a certain cryptocurrency in this case. But the latter is the general idea, the detail will be delved into in further sections of this dissertation.

\subsection{Fama Mac-Beth Regressions}
Shifting the focus to the empirical tests, the Fama Mac-Beth two-step regression is a commonly used technique in empirical finance for determining parameter estimates in asset pricing models. The technique calculates the betas and risk premiums for all risk factors believed to influence asset prices. The fundamental concept of the regression method is to predict the returns of assets by analyzing their factor exposures or characteristics that mirror exposure to a risk factor in each period.

The approach involves two consecutive stages. Betas are calculated separately for each factor and portfolio in the initial step. After that, the next step includes examining all portfolios as a whole, but separately for each time period. This method enables the analysis of how portfolio returns correspond with the betas identified in the first stage, resulting in the detection of risk premiums for each factor over different time frames. In the end, a compilation of risk premiums linked to each factor in the model is obtained.
\subsection{Gibbons, Ross, and Shanken Test}
Lastly, the statistical test outlined in \parencite{GRS1989} serves to assess the precision of asset pricing models. It is particularly employed to scrutinize whether the expected returns of a set of portfolios can be explained by their exposure to a common set of risk factors. 

Employing this test will facilitate the examination of the stated hypothesis, with a crucial emphasis on not rejecting the null hypothesis.







\cleardoublepage

\chapter{Methodological Development}
\label{c4}
\todo[inline]{I need to establish: Type of study, Period and place where the research was done, Universe and sample, Methods, Variable selection, Assumptions, Procedures, Methods of data collection.}

As it was already mentioned in previous sections of this dissertation, the investigation corresponds to an empirical finance study of a model that is derived from a theoretical formulation that was already presented in \ref{c322}. The investigation was developed in the year 2024, in \textit{Universidad de los Andes} from \textit{Santiago, Chile}. In the following sections more detail will be presented as to every step showcased in the \nameref{32} relating the research itself.

\section{Mathematical Formulation}
Although the optimization problem was already presented in \ref{c322}, it was the following.
 \begin{equation}
 	\max_{\bm{n}_{j}}E\left[U(w_{j})\right]\;.
 \end{equation}
 Subject to:
 \begin{flalign}
 	w_{j} &= \bm{n}_{j}^{\intercal}\bm{x} + n^{f}_{j}\;,&&\\
 	\Bar{w}_{j} &= \bm{n}_{j}^{\intercal}\bm{P} + n^{f}_{j}P_{f}\;.&&
 \end{flalign}
Where in summary the variables represent:
\begin{itemize}
	\item $w_{j}$ and $\Bar{w}_{j}$ are the terminal and initial wealth for investor type $j$.
	\item $\bm{n}_j$ is the vector representing the amount investor type $j$ purchases in each of $N$ cryptocurrencies.
	\item $n^{f}_{j}$ the number of risk-free discount bonds with unit payoff purchased by investor type $j$.
	\item $\bm{P}$ is the vector of cryptocurrency prices, and $P_f$ is the price of the discount bond.
\end{itemize}
An important aspect is that this model derivation is done originally in \parencite{luo2017social}. The model formulated in this dissertation is based on said paper, but in this case it is applied to cryptocurrencies. Also, in the paper the detail of the math is given, but there are some steps that are not shown in a level of detail that allows for a full understanding of the process, so in those cases mathematical intuition was required to obtain the required results in order to derive the model. Said steps will be mentioned in this section.

\subsection{Math Detail}
For the model there are two types of investors: the unrestricted investors ($U$), and the restricted investors ($R$). In this case, the restricted investors invest solely in cryptocurrencies that hold a dominant position in terms of popularity and market capitalization.

In the traditional Capital Asset Pricing Model, the unrestricted investor fully consumes terminal wealth, with $w_{U}$ being terminal wealth of the unrestricted investor. For the said investor, the problem is as follows,
\begin{gather}
	\max_{\bf{n}_{U}}\mathrm{E}\left[U(w_{U})\right] \label{objective}\;.
\end{gather}
\begin{flalign}
	&\text{Subject to:}\nonumber&&\\
	&w_{U} = \boldsymbol{n^{\intercal}}_{U}\bm{x} + n^{f}_{U} \label{final wealth}\;,&&\\
	&\Bar{w}_{U} = {\bm{n^{\intercal}}_{U}\bm{P}} + n^{f}_{U} P_{f}\label{initial wealth}\;.
\end{flalign}
The following step was not detailed in the paper. From \eqref{initial wealth}, the following conclusion can be drawn,
\begin{equation*}
	n^{f}_{U}= \frac{1}{P_{f}}\left(\Bar{w}_{U} - \bm{n^{\intercal}}_{U}\bm{P}\right)\;.
\end{equation*}
Then, substituting the expression in \eqref{final wealth},
\begin{equation}
	w_{U} = \bm{n^{\intercal}}_{U}\bm{x} + \Bar{w}_{U}\frac{1}{P_{f}} -\bm{n^{\intercal}}_{U}\underbrace{\frac{\bm{P}}{P_{f}}}_{\bm{p}} = \frac{\Bar{w}_{U}}{P_{f}} + \bm{n^{\intercal}}_{U}(\bm{x}-\bm{p}) \label{wealth cons.}
\end{equation}
Substituting \eqref{wealth cons.} in \eqref{objective}, and computing the derivative.
\begin{equation*}
	\begin{split}
		\frac{d\mathrm{E}}{dw_{U}} &= \mathrm{E}\left[U'(w_{U})(\bm{x}-\bm{p})\right]=0\;.
	\end{split}
\end{equation*}
Which corresponds to the first order condition. Now, taking into account that $\bm{x}\sim\mathcal{N}(\Bar{\bm{x}},\bm{\Sigma})$, which is that the payoff vector is multivariate normally distributed, applying the definition of covariance (see appendix \ref{app: cov-def}) the following expression can be deduced, which with the lemma application, were not detailed in the paper,
\begin{equation*}
	\begin{split}
		\mathrm{E}\left[U'(w_{U})(\bm{x}-\bm{p})\right] &= E\left[(U'(w_U)-E\left[U'(w_U)\right])(\bm{x}-\bm{p}-E\left[\bm{x}-\bm{p}\right])\right] + E\left[U'(w_U)\right]E\left[\bm{x}-\bm{p}\right]\;,\\
		&= E\left[(U'(w_U)-E[U'(w_U)])(\bm{x}-\bm{\Bar{x}})\right]+ E\left[U'(w_U)\right](\bm{\Bar{x}}-\bm{p})\;,\\
		&= E\left[U'(w_U)(\bm{x}-\Bar{\bm{x}})-E[U'(w_U)](\bm{x}-\bm{\Bar{x}})\right]+ E\left[U'(w_U)\right](\bm{\Bar{x}}-\bm{p})\;,\\
		&= E\left[U'(w_U)(\bm{x}-\Bar{\bm{x}})\right]-E[U'(w_U)]E\left[\bm{x}-\bm{\Bar{x}}\right]+ E\left[U'(w_U)\right](\bm{\Bar{x}}-\bm{p})\;,\\
		\mathrm{E}\left[U'(w_{U})(\bm{x}-\bm{p})\right] &= E\left[U'(w_U)(\bm{x}-\bm{\Bar{x}})\right]+ E\left[U'(w_U)\right](\bm{\Bar{x}}-\bm{p})=0\;.
	\end{split}
\end{equation*}
Then the following equality can be defined,
\begin{equation*}
	-E\left[U'(w_U)(\bm{x}-\bm{\Bar{x}})\right]= E\left[U'(w_U)\right](\bm{\Bar{x}}-\bm{p})
\end{equation*}
Applying the lemma in appendix \ref{app: steins lemma},
\begin{equation}
	\begin{split}
		-E\left[U''(w_U)\right]\bm{\Sigma}\bm{n}_{U}&=E\left[U'(w_U)\right](\bm{\Bar{x}}-\bm{p})\;,\\
		\bm{\Bar{\bm{x}}}-\bm{p} &= \frac{-E\left[U''(w_U)\right]}{E\left[U'(w_U)\right]}\bm{\Sigma}\bm{n}_U\;,\\
		\bm{\Bar{\bm{x}}}-\bm{p} &= \theta_U\bm{\Sigma}\bm{n}_U\;.
	\end{split}
	\label{eq:unrestricted}
\end{equation}
Where $\theta_{U}={-E\left[U''(w_U)\right]}/{E\left[U'(w_U)\right]}$ is analogous to absolute risk aversion\footnote{Tendency of individuals to prefer outcomes with low uncertainty over those with high uncertainty, even if the average outcomes of the latter is equal to or higher in monetary value that the more certain outcome \parencite{pratt1964risk}.}, which depends on the initial wealth of investor $U$ and other model. $\bm{\Sigma}$ is the covariance matrix for risky asset payoffs and $\Bar{\bm{x}}$ the expected payoffs of risky assets.

For investor type $R$ the problem is of similar nature,
\begin{gather}
	\max_{\bf{n}_{R}}\mathrm{E}\left[U(w_{R})\right]\;. \label{objective R}
\end{gather}
\begin{flalign}
	&\text{Subject to:}\nonumber&&\\
	&w_{R} = \boldsymbol{n^{\intercal}}_{R}\bm{x} + n^{f}_{R} \label{final wealth R}\;,&&\\
	&\Bar{w}_{R} = {\bm{n^{\intercal}}_{R}\bm{P}} + n^{f}_{R} P_{f}\;.\label{initial wealth R}
\end{flalign}
Where $\bm{n}_{R}$ is the vector of the shares of cryptocurrencies that investor $R$ purchases that comply with their preferences. Then, following the same procedure as before.
\begin{equation}
	\theta_{R}\bm{\Sigma}_{P}\bm{n}_{R}=\Bar{\bm{x}}_{P} - \bm{p}_{P}\;.\label{non_sin_res}
\end{equation}
Where the matrix of asset payoff covariances is partitioned into popular ($P$) and non-popular ($N$) cryptocurrencies.
\begin{equation}
	\bm{\Sigma} = \begin{bmatrix}
		\bm{\Sigma}_{P} & \bm{\Sigma}_{PN}\\
		\bm{\Sigma}_{NP} & \bm{\Sigma}_{N}
	\end{bmatrix}
	\label{covariance_matrix_gen}
\end{equation}
Where $\bm{\Sigma}_{N}$ represents the payoff covariance of all cryptocurrencies that are ``non-popular'' or have small market capitalization, and $\Bar{\bm{x}}_{N}$ and $\bm{p}_N$ are the vectors of mean payoffs and prices, respectively, of the ``non-popular'' cryptocurrencies.

Assuming $q_{U}$ investors of type $U$ and $q_R$ investors of type $R$, the demand for cryptocurrencies may be obtained and set equal to the exogenous supply of cryptocurrencies $\Bar{\bm{n}} = \left(\Bar{\bm{n}}_N, \Bar{\bm{n}}_P\right)^{\intercal}$, and to zero for the risk-free asset, yielding the conditions for market equilibrium.
\begin{equation}
	\Bar{\bm{n}} = q_{U}\bm{n}_{U} + q_{R}\bm{n}_{R},\quad 0 = q_{U}n^{f}_{U} + q_{R}n^{f}_{R}\;.\label{market eq.}
\end{equation}
Reorganizing equations \eqref{non_sin_res} and \eqref{eq:unrestricted} yields the following,
\begin{equation*}
	\bm{n}_{U} = \left(\theta_{U}\bm{\Sigma}\right)^{-1}(\Bar{\bm{x}} - \bm{p}),\quad \bm{n}_{R} = \left(\theta_{R}\bm{\Sigma}_{P}\right)^{-1}(\Bar{\bm{x}}_{P} - \bm{p}_{P})\;. 
\end{equation*}
The following step was not detailed. Note that $\bm{n}_R$ can be represented in the following form,
\begin{equation*}
	\bm{n}_R = \theta^{-1}_{R}\begin{bmatrix}
		\bm{\Sigma}^{-1}_{P} & 0\\
		0 & 0
	\end{bmatrix}
	(\Bar{\bm{x}}-\bm{p}) = \begin{bmatrix}
		\bm{I}\\
		0
	\end{bmatrix}
	\left(\bm{\Sigma}_P\theta_R\right)^{-1}\begin{bmatrix}
		\bm{I} & 0
	\end{bmatrix}
	\left(\bm{\Bar{x}}-\bm{p}\right)\;.
\end{equation*}
Substituting in \eqref{market eq.} yields the following,
\begin{equation}
	\bm{\Bar{n}}=\left(\left(\bm{\Sigma}\theta_U/q_U\right)^{-1}+\begin{bmatrix}
		\bm{I}\\
		0
	\end{bmatrix}
	\left(\bm{\Sigma}_P \theta_R/q_R\right)^{-1}\begin{bmatrix}
		\bm{I} & 0
	\end{bmatrix}\right)\left(\bm{\Bar{\bm{x}}}-\bm{p}\right)\;.
	\label{eq:dem_exogena}
\end{equation}
From where we want to isolate the expression $\bm{\Bar{x}}-\bm{p}$, then is necessary to compute the inverse of the expression in parenthesis. The latter can be done using an identity \parencite{Soderstrom2002} that says the following. Given matrices $\bm{X}_1, \bm{X}_2, \bm{X}_3 \text{ y } \bm{X}_4$, with $\bm{X}_1$, $\bm{X}_4$ having an inverse, the following equality is satisfied.
\begin{equation}
	\left(\bm{X}^{-1}_1 + \bm{X}_2\bm{X}^{-1}_{4}\bm{X}_3\right)^{-1} = \bm{X}_1 + \bm{X}_1\bm{X}_2\left(\bm{X}_4 + \bm{X}_3\bm{X}_1\bm{X}_2\right)^{-1}\bm{X}_3\bm{X}_1\;.
	\label{eq:id_sodes}
\end{equation}
This step and the subsequent one were not detailed. Substituting the terms in \eqref{eq:id_sodes}, yields the following,
\begin{equation*}
	\begin{split}
		& \left(\left(\bm{\Sigma}\theta_U/q_U\right)^{-1}+\begin{bmatrix}
			\bm{I}\\
			0
		\end{bmatrix}
		\left(\bm{\Sigma}_P \theta_R/q_R\right)^{-1}\begin{bmatrix}
			\bm{I} & 0
		\end{bmatrix}\right)^{-1}\\ 
		&= \bm{\Sigma}\theta_U/q_U -\bm{\Sigma}\theta_U/q_U\begin{bmatrix}
			\bm{I}\\
			0
		\end{bmatrix}
		\left(\bm{\Sigma}_P \theta_R/q_R + \begin{bmatrix}
			\bm{I} & 0
		\end{bmatrix}
		\bm{\Sigma}\theta_U/q_U\begin{bmatrix}
			\bm{I}\\
			0
		\end{bmatrix}
		\right)^{-1}
		\begin{bmatrix}
			\bm{I} & 0
		\end{bmatrix}
		\bm{\Sigma}\theta_U/q_U\;,\\
		&= \bm{\Sigma}\theta_U/q_U - \bm{\Sigma}\theta_U/q_U\begin{bmatrix}
			\bm{I}\\
			0
		\end{bmatrix}
		\left(\bm{\Sigma}_P \theta_R/q_R + \bm{\Sigma}_P\theta_U/q_U\right)^{-1}\begin{bmatrix}
			\bm{I} & 0
		\end{bmatrix}
		\bm{\Sigma}\theta_U/q_U\;,\\
		&= \theta_U/q_U\left(\bm{\Sigma} - \frac{\theta_U/q_U}{\theta_U/q_U + \theta_R/q_R}\bm{\Sigma}\begin{bmatrix}
			\bm{\Sigma}^{-1}_{P} & 0\\
			0 & 0
		\end{bmatrix}
		\bm{\Sigma}
		\right)\;.
	\end{split}
\end{equation*}
Then, substituting the expression in \eqref{eq:dem_exogena} yields the following,
\begin{equation}
	\begin{split}
		(\bm{\Bar{x}}-\bm{p})&=\theta_{U}/q_{U}\left(\bm{\Sigma} - \frac{\theta_U/q_U}{\theta_U/q_U + \theta_R/q_R}\bm{\Sigma}\begin{bmatrix}
			\bm{\Sigma}^{-1}_{P} & 0\\
			0 & 0
		\end{bmatrix}
		\bm{\Sigma}
		\right)\bm{\Bar{n}}\;,\\
		&=\theta_{U}/q_{U}\left(\bm{\Sigma} - \frac{\theta_U/q_U}{\theta_U/q_U + \theta_R/q_R}\bm{\Sigma}\begin{bmatrix}
			\bm{I} & \bm{\Sigma}^{-1}_{P}\bm{\Sigma}_{PN}\\
			0 & 0
		\end{bmatrix}
		\right)\bm{\Bar{n}}\;,\\
		&= \theta_{U}/q_{U}\left(\bm{\Sigma}\bm{\Bar{n}} - \frac{\theta_U/q_U}{\theta_U/q_U + \theta_R/q_R}\bm{\Sigma}\begin{bmatrix}
			\bm{\Bar{n}}_{N}+\bm{\Sigma}^{-1}_{P}\bm{\Sigma}_{PN}\bm{\Bar{n}}_{P}\\
			0 
		\end{bmatrix}
		\right)\;,\\
		&= \theta_{U}/q_{U}\left(\bm{\Sigma}\bm{\Bar{n}} -\frac{\theta_U/q_U}{\theta_U/q_U + \theta_R/q_R}\bm{\Sigma}\bm{\Bar{n}}+ \frac{\theta_U/q_U}{\theta_U/q_U + \theta_R/q_R}\bm{\Sigma}\begin{bmatrix}
			-\bm{\Sigma}^{-1}_{P}\bm{\Sigma}_{PN}\bm{\Bar{n}}_{P}\\
			\bm{\Bar{n}}_{P} 
		\end{bmatrix}
		\right)\;,\\
		&= \theta_{U}/q_{U}\left(\frac{\theta_R/q_R}{\theta_U/q_U + \theta_R/q_R}\bm{\Sigma}\bm{\Bar{n}}+ \frac{\theta_U/q_U}{\theta_U/q_U + \theta_R/q_R}\bm{\Sigma}\begin{bmatrix}
			-\bm{\Sigma}^{-1}_{P}\bm{\Sigma}_{PN}\bm{\Bar{n}}_{P}\\
			\bm{\Bar{n}}_{P} 
		\end{bmatrix}
		\right)\;,\\
		&= \left(\frac{1}{q_U/\theta_U + q_R/\theta_R}\bm{\Sigma}\bm{\Bar{n}}+ \frac{1}{q_U/\theta_U + q_R/\theta_R}\frac{q_{R}/\theta_R}{q_U/\theta_U}\bm{\Sigma}\begin{bmatrix}
			-\bm{\Sigma}^{-1}_{P}\bm{\Sigma}_{PN}\bm{\Bar{n}}_{P}\\
			\bm{\Bar{n}}_{P} 
		\end{bmatrix}
		\right)\;,\\
		&= \frac{1}{q_U\Bar{w}_U/\rho_U + q_R\Bar{w}_R/\rho_R}\bm{\Sigma}\bm{\Bar{n}}+ \frac{1}{q_U\Bar{w}_U/\rho_U + q_R\Bar{w}_R/\rho_R}\frac{q_{R}\Bar{w}_R/\rho_R}{q_U\Bar{w}_U/\rho_U}\bm{\Sigma}\begin{bmatrix}
			-\bm{\Sigma}^{-1}_{P}\bm{\Sigma}_{PN}\bm{\Bar{n}}_{P}\\
			\bm{\Bar{n}}_{P} 
		\end{bmatrix}
		\;,\\
		&=\gamma\bm{\Sigma}\bm{\Bar{n}} + \delta\bm{\Sigma}\bm{\Bar{n}}_K\;.
	\end{split}
	\label{eq:id-boicott}
\end{equation}
Where $\bm{\Bar{n}}_K$ represents the known cryptocurrency portfolio. Now, \eqref{eq:id-boicott} must be converted into an expression for expected returns rather than expected net payoffs. Given that $P_f=1/(1+r_f)$, the following can be defined,
\begin{equation*}
	(1+r^{s}_{i}) = \frac{x_i}{P_i} \Leftrightarrow x_i - \frac{P_i}{P_f}=P_i(1+r^{s}_{i}) - P_i(1+r_f)=P_i(r^{s}_{i}-r_f)\;.
\end{equation*}
Then, defining the excess return as $r_i = r^{s}_{i} - r_{f}$, and given that in Equation \eqref{eq:id-boicott} the expression to the left of the equality is represented as an average, it follows that $\mu_{i} = \mu^{s}_{i} - r_{f}$. In addition, since $1 + r^{s}_{i} = x_i / P_i$, the covariance matrix for the payoffs of the cryptocurrencies $\bm{\Sigma}$ can be represented in terms of the returns as $\sigma_{ij} = \Sigma_{ij} / P_{i} P_{j}$. Thus, for a specific element of Equation \eqref{eq:id-boicott}, it can be stated that,
\begin{equation}
	\begin{split}
		P_i \mu_i&=\gamma \Sigma_{im} + \delta\Sigma_{ip}\\
		\mu_{i} &= \gamma P_m \sigma_{im} + \delta P_p \sigma_{ip}
	\end{split}
	\label{eq:mean-returns}
\end{equation}
Where $m$ represents the market, $P_m = q_m \Bar{w}_{M} = q_U \Bar{w}_{U} + q_R \Bar{w}_{R}$ is the cost of the market portfolio, and $P_p$ is the cost of the popular portfolio. Now, given \eqref{eq:mean-returns}, $\mu_{m}$ and $\mu_{p}$ can be defined, which correspond to the mean returns of the market and popular portfolios, respectively.
\begin{equation*}
	\mu_{m} = \gamma P_m \sigma^{2}_{m} + \delta P_p \sigma_{mp} \quad; \quad \mu_{p} = \gamma P_m \sigma_{mp} + \delta P_p \sigma^{2}_{p}\;.
\end{equation*}
Solving the system of equations for $\gamma P_m$ y $\delta P_p$ yields the following,
\begin{equation*}
	\delta P_p = \frac{\sigma_{mp}\mu_{m}-\sigma^{2}_{m}\mu_{p}}{\sigma^{2}_{mp}-\sigma^{2}_{p}\sigma^{2}_{m}}\quad ; \quad \gamma P_m = \frac{\sigma_{mp}\mu_{p}-\sigma^{2}_{p}\mu_{m}}{\sigma^{2}_{mp}-\sigma^{2}_{p}\sigma^{2}_{m}}\;.
\end{equation*}
This step was not detailed. Substituting in \eqref{eq:mean-returns} yields,
\begin{equation}
	\begin{split}
		\mu_{i} &= \frac{\sigma^{2}_{m}\sigma_{ip}-\sigma_{mp}\sigma_{im}}{\sigma^{2}_{p}\sigma^{2}_{m}-\sigma^{2}_{mp}}\mu_{p} + \frac{\sigma^{2}_{p}\sigma_{im}-\sigma_{mp}\sigma_{ip}}{\sigma^{2}_{p}\sigma^{2}_{m}-\sigma^{2}_{mp}}\mu_{m}\;,\\
		\mu_{i}&= \beta_{ip}\mu_{p} + \beta_{im}\mu_{m}\;.\\  
	\end{split}
\end{equation}
Where $\beta_{ib}$ and $\beta_{ip}$ are the population values of the slope estimates for a linear regression of the return of asset $i$ on the market portfolio return and the popular portfolio return.

\subsubsection{Some comments}
An attempt was made to modify the original derivation process that uses two type of investor, by adding a third one. But the issue with this is that the step in which \parencite{Soderstrom2002} identity is used, there are problems respecting the amount of matrices needed to use said identity. Assuming there are three types of investors $U$, $R_1$, and $R_2$, the optimization problems do not change, but the covariance matrix does,
\begin{equation*}
	\bm{\Sigma} = \begin{bmatrix}
		\bm{\Sigma}_{P_1} & \bm{\Sigma}_{P_1 P_2} & \bm{\Sigma}_{P_1 N}\\
		\bm{\Sigma}_{P_2 P_1} & \bm{\Sigma}_{P_2} & \bm{\Sigma}_{P_2 N}\\
		\bm{\Sigma}_{N P_1} & \bm{\Sigma}_{N P_2} & \bm{\Sigma}_{N}
	\end{bmatrix}\;.
\end{equation*}
Then, the terms respecting the exogenous supply of cryptocurrencies would be as follows,
\begin{equation}
	\Bar{\bm{n}} = q_{U}\bm{n}_{U} + q_{R_1}\bm{n}_{R_1}+ q_{R_2}\bm{n}_{R_2},\quad 0 = q_{U}n^{f}_{U} + q_{R_1}n^{f}_{R_1} + q_{R_2}n^{f}_{R_2}\;\label{market-eq-3}.
\end{equation} 
Defining the following,
\begin{equation*}
	\bm{n}_{U} = \left(\theta_{U}\bm{\Sigma}\right)^{-1}(\Bar{\bm{x}} - \bm{p}),\quad \bm{n}_{R_1} = \left(\theta_{R_1}\bm{\Sigma}_{P_1}\right)^{-1}(\Bar{\bm{x}}_{P_1} - \bm{p}_{P_1}),\quad \bm{n}_{R_2} = \left(\theta_{R_2}\bm{\Sigma}_{P_2}\right)^{-1}(\Bar{\bm{x}}_{P_2} - \bm{p}_{P_2})\;. 
\end{equation*}
This would imply that if those therms are replaced in \eqref{market-eq-3}, three new matrices appear in the expression in which the identity presented in \parencite{Soderstrom2002} is applied. Making it impossible to apply said identity, cause it requires only four matrices, not six.

\section{Data}
As it was already mentioned on \ref{321}, the source of the data corresponds to the \textit{Yahoo Finance} library in \textit{Python}, that is used in conjunction with \textit{Yahooquery}, where the latter is utilized for the retrieval of cryptocurrency symbols. 

Regarding the universe and the sample, the initial set comprised 250 cryptocurrencies. Stablecoins were excluded from this set, resulting in a total of 226 cryptocurrencies paired with the US Dollar. An attempt was made to eliminate stablecoins using a criterion based on the standard deviation and mean of their returns. However, for accuracy, the identification and removal of these stablecoins were ultimately done manually. Conversely the sample, like it was already mentioned in \ref{321}, were 250 cryptocurrencies ordered by intraday market capitalization. The dates ranges of the data are from September, 2014 till March, 2024.

\subsection{Market Index}
A crypto market index was utilized in order to have a the market factor in the model. Said factor was the \textit{Crypto200 ex BTC Index by Solactive}, that is comprised by a volume weighted average price\footnote{Trading metric that calculates the average price of an asset based on both the trading volume and the price. It gives more weight to trades with higher volume. It provides a more accurate representation of the average price considering the trading activity.}\todo{Add source.} on the top 200 cryptocurrencies except for Bitcoin and stablecoins, by market capitalization, in USD.

One important detail is that the data for this index has been available only from 2019 onward. This limitation affected the sample size, necessitating a reduction to conduct the respective regressions.

\subsection{Return computation}
All the returns in this investigation correspond to weekly returns. To compute them, the market index data was retrieved first, as it had fewer available dates than the cryptocurrency data. Weekly values of the market index were collected, and the returns were subsequently calculated.

Having the dates of the market returns, for the calculations of the cryptocurrency returns, daily data was used, and the dates were filtered in order for them to match the ones of the market data. Then, with the filtered dates, the weekly returns were computed for each cryptocurrency in the dataset.

\subsection{``Popular'' Factor Estimation}
\label{c423}
In order to build the ``popular'' factor that is derived in the model, a portfolio comprised of these type of cryptocurrencies was built (see table \ref{tab:cryptos} for the detail). In the other hand, the ``not popular'' portfolio was built using the rest of the cryptocurrencies that were not in the ``popular'' portfolio. Then, the returns for each portfolio were calculated using a value weighted fashion, with their respective market capitalization's.

One aspect that is important to detail that also was used in further stages of the data handling, was that the dates in which the cryptocurrencies have data vary depending on each one. The latter because there are cryptocurrencies that are newer than others. This affects directly the computation of portfolio returns because there will be dates were not all the cryptocurrencies in the portfolio will have available returns.

So, for the portfolio returns, the weights are computed for each date based on the availability of returns for the cryptocurrencies in the portfolio on that date. This method was used for all the portfolio computations on this investigation.

After calculating the returns for each portfolio, statistical analysis was conducted. First, a histogram of the returns was created to observe whether the distribution resembled a normal distribution. Analyzing figure \ref{fig:histogram-of-portfolio-returns}, the resemblance is quite similar to a normal distribution.
\begin{figure}[h!]
	\centering
	\includegraphics[width=0.95\linewidth]{"Histogram of portfolio returns"}
	\caption{Histogram of portfolio returns for ``popular'' and ``not popular'' cryptocurrencies.}
	\label{fig:histogram-of-portfolio-returns}
\end{figure}

Then, t-student hypothesis test was made, in order to check if there is a significant difference in the returns of both portfolios, because in otherwise, the idea of a ``popular'' factor would loose credibility in explaining the cross section of returns.

The t-statistic in a t-test indicates how many standard errors the sample mean is from the sample mean of another group. The sign of the t-statistic can also indicate the direction of the difference, if one exists. In this case, a positive sign suggests that the mean returns of the ``not popular" portfolio are greater than those of the ``popular" portfolio, while a negative sign suggests the opposite. One important detail is that for this test the whole date range was used that is available in the cryptocurrency dataset, which is large that the date range of the market index data.
\begin{table}[h!]
	\centering
	\captionsetup{skip=0.5\baselineskip}
	\caption{T-Test Results.}
	\begin{tabular}{|c|c|}
		\hline
		\textbf{Statistic} & \textbf{P-value} \\ \hline
		1.71 & 0.088 \\ \hline
	\end{tabular}
	\label{tab:ttest-results}
\end{table}

The null hypothesis in this case is that there is no significant difference in portfolio returns, while the alternative hypothesis suggests otherwise. From table \ref{tab:ttest-results}, it can be concluded that there is sufficient statistical evidence to reject the null hypothesis in favor of the alternative. Specifically, the results indicate that the mean portfolio return of the ``not popular" portfolio is greater than that of the ``popular" portfolio.

The last ``test'' that was done corresponded to the computation of the mean portfolio returns of the whole sample, using all the dates available in the cryptocurrency data.
\begin{table}[h!]
	\centering
	\captionsetup{skip=0.5\baselineskip}
	\caption{Mean Portfolio Returns.}
	\begin{tabular}{|c|c|}
		\hline
		\textbf{Portfolio} & \textbf{Mean Return (\%)} \\ \hline
		Not Popular & 10.23\% \\ \hline
		Popular & 1.54\% \\ \hline
	\end{tabular}
	\label{tab:mean-portfolio-returns}
\end{table}
From table \ref{tab:mean-portfolio-returns}, it can be seen that there is an important difference in mean returns, when taking into consideration the complete date range.

With all the necessary statistical tests completed, the process of constructing the ``popular" factor could commence. Initially, a zero investment portfolio was constructed utilizing the difference in returns between the``popular" and ``not popular" portfolios. Subsequently, a regression was performed, with the market returns serving as the independent variable and the zero investment portfolio returns as the dependent variable. Ultimately, the estimation of the ``popular" factor corresponds to the residual error of this regression model.

\section{Portfolio Building}
In \parencite{fama2004capital} it is explained that while CAPM can be a useful tool, it often performs better when applied to portfolios rather than individual assets, mainly because of the diversification of idiosyncratic risk. For that reason, portfolios of cryptocurrencies were built in order to test the validity of the model.

Tu build said portfolios, regressions were estimated for every cryptocurrency, using the ``popular'' factor as the independent variable, and the returns of an individual cryptocurrency as the dependent variable. So, for every cryptocurrency, a beta was estimated for the ``popular'' factor.

Using those estimated betas, all the cryptos were arranged from higher to lower value. Then, the portfolios were formed choosing from said list in descending order, based on a certain number of cryptocurrencies for each one.

A really important detail is the one mentioned in section \ref{c423} that is related to the forming of portfolios and the availability of returns on certain dates. In this part, that phenomena impacted directly on the number of cryptos that every portfolio has, because it had to be in a way that all the portfolios could have returns in all the range of dates so that regressions could be done later on. A lot of alternatives were tested, but the number that allowed for the regressions was a total of 26 cryptocurrencies in each portfolio.

Although said quantity meant that there was one portfolio with less cryptos than others, the latter did not poise a problem to carry on with the regressions. In detail portfolios 1 through 8 had 26 cryptocurrencies, and the ninth portfolio had 18.

Finally, for each portfolio, the respective returns were computed using value weighted and equally weighted methods.

\section{Fama Mac-Beth Regressions}
In section \ref{fama regressions} a general overview of the methodology was presented, but in practice, another approach was taken that is similar to the one already mentioned. In this case, the first pass is the GRS test presented on section \ref{GRStest}, where a p-value is obtained to determine if the null hypothesis is rejected or not, it is important to remember that the objective is not to reject said hypothesis.

The second pass consisted of two consecutive stages, as detailed in section \ref{fama regressions}. First, one regression is estimated for the average returns, were an adjusted R-squared is obtained as well as the factor loadings for every factor with its respective t statistic value. Second, a cross-sectional regression is estimated for every week, from were an average R-squared is obtained with the estimations of the factors loadings and their respective t statistics. From this step, coefficient estimates are obtained for every factor and the intercept, and also an average R-squared, which is normally the one that is used for empirical studies.

The ideal outcome in this section is to achieve a high adjusted R-squared, with an intercept that is not statistically significant, and statistically significant factor coefficients.







 
\cleardoublepage

\chapter{Results}
\label{c5}
For the testing of the model the methods presented in previous chapters were utilized, and the results obtained will be reviewed. The Fama Mac-Beth methodology was made on various time periods to test if the model presented different results based on that factor. Although there were results on various time periods, in this chapter the focus will be in the results obtained using all the dataset.
\begin{table}[h!]
	\centering
	\captionsetup{skip=0.5\baselineskip}
	\caption{Fama Mac-Beth results for all the dataset}
	\begin{tabular}{|c|c|}
		\hline
		\textbf{Factor} & \textbf{CAPM-model} \\ \hline
		\multicolumn{2}{|c|}{\textbf{For average returns}} \\ \hline
		Intercept & 0.0213 \\ 
		t-stat & (3.3324) \\ \hline
		Popular & 0.0044 \\ 
		t-stat & (1.5881)\\ \hline
		Market & -0.0973 \\
		t-stat & (-2.8945)\\ \hline
		\textbf{Adjusted R-squared} & 0.4458 \\ \hline
		\multicolumn{2}{|c|}{\textbf{Cross-sec reg. for every month}} \\ \hline
		Intercept & 0.0213 \\ 
		t-stat & (1.8669) \\ \hline
		Popular & 0.0044 \\ 
		t-stat & (1.9031)\\ \hline
		Market & -0.0973 \\
		t-stat & (-1.4573)\\ \hline
		\textbf{Average Adjusted R-squared} & 0.3338 \\ \hline
		\multicolumn{2}{|c|}{\textbf{GRS}} \\ \hline
		F-GRS (95\%) & 1.3848 \\ \hline
		F-GRS (99\%) & 1.5810 \\ \hline
		p-value & 0.0110 \\ \hline
	\end{tabular}
	\label{table: Results-all-data}
\end{table}

On table \ref{table: Results-all-data} are the results obtained using the Fama Mac-Beth methodology. From these, in relation to the desired outcomes mentioned on section \ref{c44}, only one of them is met, which is a not statistically significant intercept, in the case of cross sectional regressions for every month. So, the model does not have statistically significant factors, and has a low R-squared, meaning that it is failing to explain the variability of returns in the data that was used for this opportunity.

The lack of performance of the model may be because important factors like size and momentum, which impact returns, were not included. The existing variables do not have statistical importance, indicating that they do not account for a significant amount of the variance in cryptocurrency returns. This means that the model does not accurately represent critical market dynamics. Taking into account important factors such as market capitalization and price trends is crucial for enhancing the model's precision and explanatory capability. Without them, the model does not manage to explain a great portion of the variability in the returns of cryptocurrencies.

Another reason that might explain the lack of performance of the model is the data. One aspect could be that the quality and quantity of the data is affecting the results. This, because in studies like \parencite{liu2022common}, Coinmarketcap.com is used for the data, which is the leading source of cryptocurrency price and volume data. It aggregates information from over 200 major exchanges and provides daily data on opening closing, high, and low prices as well as volume and market capitalization (in dollars) for most of the cryptocurrencies. 

To put into perspective the sample in said study included 1,827 coins; and in this dissertation the API used, allowed a maximum of 250. So, the lesser amount of cryptocurrencies can be affecting the results, because it impacts the number of portfolios that can be formed for the Fama Mac-Beth methodology.

Continuing with the data, another aspect is that the sample period could be affecting the results, meaning that the model might perform better during certain periods. To test this hypothesis, the model was evaluated on various time periods, particularly on dates when the cryptocurrency market was in a bull state, following the findings of \parencite{abugri2024bullbear}\todo{Check this citation}. The results for these time periods can be found in Appendix \ref{app: bull-res}.

Although there is a difference in the model's results using the Fama-MacBeth methodology, none of the time periods fulfill all the requirements presented in section \ref{c44} to obtain an ideal outcome. Additionally, it may seem that the model results are being selected based on time periods that could generate favorable outcomes. However, if the model effectively explained the variability in cryptocurrency returns, selecting a specific time frame for testing would not be necessary.

Another important aspect that could explain the lack of performance is the theoretical foundation of the model. When asset pricing models are tested with real data, the empirical models usually have better performances in comparison to CAPM, that has a theoretical derivation.

In \parencite{jiang2022comparison} they conclude that the CAPM does not explain the cryptocurrency market well due to the reason that said model does not take other factors into account like a size factor, which is relatively important in the cryptocurrency market as stated in \parencite{liu2022common}. Other important factors mentioned in \parencite{liu2022common} are cryptocurrency market and momentum, from which the model tested in this dissertation only includes the first one mentioned, which is the market.

Continuing the same idea, \parencite{Shen2020} does a comparison between a three factor model, that contains the market, size and reversal factors, using the cryptocurrency CAPM as a benchmark. The findings are that the three factor model mentioned earlier strongly dominates the cryptocurrency CAPM.

The cases presented earlier, show strong evidence that the performance of empirical models that include several factors, usually outperforms the CAPM. So, as it was already mentioned, evidence exists that usually CAPM does not perform well when tested with real data, leading to believe that, although the derivation of the model is very interesting because the theory behind is well-founded, it does not correlate with good performance.

To conclude, there is no correct explanation as to why the model performed as it did, but the reasons talked about in this chapter may be some of the most relevant, to help explain the lack of performance. Furthermore, it might have been a combination of the reasons, or simply one could be more impactful, but the recognition of said reasons will help provide insights in upcoming chapters to help future studies that want to derive a model with theoretical foundations, to better the results of this dissertation.





\cleardoublepage

\chapter{Conclusions and Recommendations}
\label{c6}

\section{Conclusions}
The results of this study lead to several generalizable scientific conclusions that respond directly to the research objectives and scientific questions. These conclusions are interrelated with the analysis and discussion of the results, deriving directly from them.

At first, in regards to the overall goal, the initial step was accomplished by creating a portfolio model based on the ideas of Markowitz. Yet, as shown in Section \ref{c5}, the model's lackluster performance limits its capability to improve comprehension and endorse well-informed investment choices within the cryptocurrency market.

Going into the specific objectives, like it was already mentioned, the derivation of a theory backed model was successful, which was inspired in the detail presented in \parencite{luo2017social}. Said detail contains all the related mathematical equations and fundamental principles associated to CAPM, which are mentioned in the formulation itself and complemented in the appendix of this dissertation.

The collected data was analyzed, and statistical tests were conducted to validate the computed returns for the sampled cryptocurrencies and to perform a preliminary evaluation of the mean returns of the popular portfolio. Although these tests yielded positive results regarding the returns, the model's performance did not align with these findings later on. Identified data limitations may have impacted the results. Other studies using larger datasets generally obtained better outcomes, highlighting the importance of the data sample in empirical finance research.

The Fama-MacBeth methodology and the GRS test both showed consistently weak results in terms of the model's ability to explain. Various explanations were suggested for these results, such as model limitations, key missing elements, and the effectiveness of empirical models. While one reason alone may not completely explain the model's performance, when taken together, they likely account for a substantial portion of the results. 

This indicates that theoretically based models, while they might appear strong as to explaining return variability due to the theory behind, frequently do not perform as well as empirical models that consider more important factors when analyzed with actual data.

Finally, due to the lack of performance of the model, no practical implications were presented in order for individual investors and financial institutions to help obtain a better understanding of the factors that drive the returns of the cryptocurrency market.


\section{Recommendations}
The recommendations derived from these conclusions are concrete and closely related to them, aiming to improve future research and model development.

Starting with the data, having a comprehensive, high-quality dataset, such as those provided by Coinmarketcap.com, is crucial for research studies of this nature. Such datasets help eliminate potential factors that could negatively impact the performance of the tested model. However, it is important to note that these datasets are expensive and, due to the nature of this dissertation, were not accessed for this reason.

Although a mathematical derivation seems to be ideal for a model, it is a complex path to follow, and will most probably yield bad outcomes when tested later. But if this is the preferred path for a future research study, the recommendation is to modify the mathematical formulation with a factor that is already tested, and it has studies that prove the ability of these in explaining the variability in cryptocurrency returns specifically.

The previously mentioned approach will likely be very labor-intensive. Adding an extra factor does not always correlate with a successful mathematical derivation, as discussed in Section \ref{c411}. This complexity underscores the necessity of patience and persistence throughout the research process. Researchers must be prepared to reiterate their analyses and adjust their models as needed. If initial results are disappointing, it is important to systematically review each step, consider potential adjustments, and re-test the model. This iterative process is essential for refining the model and ultimately achieving more accurate and reliable outcomes.





\cleardoublepage


% Bibliography
\printbibliography[heading=bibintoc, title = {References}]



\appendix % seguido de archivos de apendices

\cleardoublepage

% Escribe 'Anexos' en el Indice General
\addcontentsline{toc}{chapter}{Appendix}

% anexo_a.tex

\chapter{Technical details, tables, and others}
\label{app_a}

\section{Covariance Definition}
\label{app: cov-def}
The definition used in the derivation of the model is the one that is presented in \parencite{Intro-prob-roussas} that states the following, assuming $X$ and $Y$ are random variables that have finite expectations,
\begin{equation}
	Cov(X,Y) = E\left[(X-E(X))(Y-E(Y)\right] = E\left[XY\right] - (E\left[X\right])(E\left[Y\right])\;.
\end{equation}

\section{Stein's Lemma}
\label{app: steins lemma}

\section{Popular Portfolio}
\begin{table}[h!]
	\centering
	\begin{tabular}{|c|c|}
		\hline
		\textbf{Name of Cryptocurrency} & \textbf{Symbol} \\ \hline
		Bitcoin & BTC-USD \\ \hline
		Ethereum & ETH-USD \\ \hline
		Binance Coin & BNB-USD \\ \hline
		Solana & SOL-USD \\ \hline
		Ripple & XRP-USD \\ \hline
		Cardano & ADA-USD \\ \hline
		Polkadot & DOT-USD \\ \hline
		Chainlink & LINK-USD \\ \hline
		Polygon & MATIC-USD \\ \hline
	\end{tabular}
	\caption{List of Popular Cryptocurrencies and their Symbols}
	\label{tab:cryptos}
\end{table}    % archivo de apendice

%\input{apendice_b}    % otro archivo de apendice

\label{end}
\end{document}

%%%%%%%%%%%%%%%%%%%%%%%%%%%%%%%%%%%%%%%%%%%%%%%%%%%%%%%%%%%%
