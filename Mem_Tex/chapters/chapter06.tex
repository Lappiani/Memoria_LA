\chapter{Conclusions and Recommendations}
\label{c6}

\section{Conclusions}
The results of this study lead to several generalizable scientific conclusions that respond directly to the research objectives and scientific questions. These conclusions are interrelated with the analysis and discussion of the results, deriving directly from them.

At first, in regards to the overall goal, the initial step was accomplished by creating a portfolio model based on the ideas of Markowitz. Yet, as shown in Section \ref{c5}, the model's lackluster performance limits its capability to improve comprehension and endorse well-informed investment choices within the cryptocurrency market.

Going into the specific objectives, like it was already mentioned, the derivation of a theory backed model was successful, which was inspired in the detail presented in \parencite{luo2017social}. Said detail contains all the related mathematical equations and fundamental principles associated to CAPM, which are mentioned in the formulation itself and complemented in the appendix of this dissertation.

The collected data was analyzed, and statistical tests were conducted to validate the computed returns for the sampled cryptocurrencies and to perform a preliminary evaluation of the mean returns of the popular portfolio. Although these tests yielded positive results regarding the returns, the model's performance did not align with these findings later on. Identified data limitations may have impacted the results. Other studies using larger datasets generally obtained better outcomes, highlighting the importance of the data sample in empirical finance research.

The Fama-MacBeth methodology and the GRS test both showed consistently weak results in terms of the model's ability to explain. Various explanations were suggested for these results, such as model limitations, key missing elements, and the effectiveness of empirical models. While one reason alone may not completely explain the model's performance, when taken together, they likely account for a substantial portion of the results. 

This indicates that theoretically based models, while they might appear strong as to explaining return variability due to the theory behind, frequently do not perform as well as empirical models that consider more important factors when analyzed with actual data.

Finally, due to the lack of performance of the model, no practical implications were presented in order for individual investors and financial institutions to help obtain a better understanding of the factors that drive the returns of the cryptocurrency market.


\section{Recommendations}
The recommendations derived from these conclusions are concrete and closely related to them, aiming to improve future research and model development.

Starting with the data, having a comprehensive, high-quality dataset, such as those provided by Coinmarketcap.com, is crucial for research studies of this nature. Such datasets help eliminate potential factors that could negatively impact the performance of the tested model. However, it is important to note that these datasets are expensive and, due to the nature of this dissertation, were not accessed for this reason.

Although a mathematical derivation seems to be ideal for a model, it is a complex path to follow, and will most probably yield bad outcomes when tested later. But if this is the preferred path for a future research study, the recommendation is to modify the mathematical formulation with a factor that is already tested, and it has studies that prove the ability of these in explaining the variability in cryptocurrency returns specifically.

The previously mentioned approach will likely be very labor-intensive. Adding an extra factor does not always correlate with a successful mathematical derivation, as discussed in Section \ref{c411}. This complexity underscores the necessity of patience and persistence throughout the research process. Researchers must be prepared to reiterate their analyses and adjust their models as needed. If initial results are disappointing, it is important to systematically review each step, consider potential adjustments, and re-test the model. This iterative process is essential for refining the model and ultimately achieving more accurate and reliable outcomes.

 

